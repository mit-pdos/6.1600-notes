% packages

\usepackage[utf8]{inputenc}
\usepackage[T1]{fontenc}
\usepackage[english]{babel}
\usepackage[final]{microtype}
\usepackage{hyperref}
\usepackage{enumitem}
\usepackage{csquotes}
\usepackage{framed}
\usepackage{xcolor}
\usepackage{tikz}

\usetikzlibrary{shapes.geometric, arrows, positioning}

%\usepackage[lambda,adversary,advantage,asymptotics,sets,landau,probability,operators,primitives]{cryptocode}
\usepackage{amsmath,amssymb,amsthm,mathtools}

\usepackage{capt-of}

% \usepackage{tikz} % Figures
\usepackage{graphicx} % Include figures

% page layout

%\setlrmarginsandblock{0.15\paperwidth}{*}{1} % Left and right margin
%\setulmarginsandblock{0.2\paperwidth}{*}{1}  % Upper and lower margin
%\checkandfixthelayout

% sections

\usepackage[capitalize]{cleveref}
% number down to subsections
\setcounter{secnumdepth}{2}

%\maxsecnumdepth{subsection} % Subsections (and higher) are numbered
%\setsecnumdepth{subsection}
\iffalse
\makeatletter %
\makechapterstyle{standard}{
  \setlength{\beforechapskip}{0\baselineskip}
  \setlength{\midchapskip}{1\baselineskip}
  \setlength{\afterchapskip}{8\baselineskip}
  \renewcommand{\chapterheadstart}{\vspace*{\beforechapskip}}
  \renewcommand{\chapnamefont}{\centering\normalfont\Large}
  \renewcommand{\printchaptername}{\chapnamefont \@chapapp}
  \renewcommand{\chapternamenum}{\space}
  \renewcommand{\chapnumfont}{\normalfont\Large}
  \renewcommand{\printchapternum}{\chapnumfont \thechapter}
  \renewcommand{\afterchapternum}{\par\nobreak\vskip \midchapskip}
  \renewcommand{\printchapternonum}{\vspace*{\midchapskip}\vspace*{5mm}}
  \renewcommand{\chaptitlefont}{\centering\bfseries\LARGE}
  \renewcommand{\printchaptertitle}[1]{\chaptitlefont ##1}
  \renewcommand{\afterchaptertitle}{\par\nobreak\vskip \afterchapskip}
}
\makeatother

\chapterstyle{standard}

\setsecheadstyle{\normalfont\large\bfseries}
\setsubsecheadstyle{\normalfont\normalsize\bfseries}
\setparaheadstyle{\normalfont\normalsize\bfseries}
\setparaindent{0pt}\setafterparaskip{0pt}

% header / footer

\makepagestyle{standard} % Make standard pagestyle

\makeatletter                 % Define standard pagestyle
\makeevenfoot{standard}{}{}{} %
\makeoddfoot{standard}{}{}{}  %
\makeevenhead{standard}{\bfseries\thepage\normalfont\qquad\small\leftmark}{}{}
\makeoddhead{standard}{}{}{\small\rightmark\qquad\bfseries\thepage}
% \makeheadrule{standard}{\textwidth}{\normalrulethickness}
\makeatother                  %

\makeatletter
\makepsmarks{standard}{
\createmark{chapter}{both}{shownumber}{\@chapapp\ }{ \quad }
\createmark{section}{right}{shownumber}{}{ \quad }
\createplainmark{toc}{both}{\contentsname}
\createplainmark{lof}{both}{\listfigurename}
\createplainmark{lot}{both}{\listtablename}
\createplainmark{bib}{both}{\bibname}
\createplainmark{index}{both}{\indexname}
\createplainmark{glossary}{both}{\glossaryname}
}
\makeatother                               %

\makepagestyle{chap} % Make new chapter pagestyle

\makeatletter
\makeevenfoot{chap}{}{\small\bfseries\thepage}{} % Define new chapter pagestyle
\makeoddfoot{chap}{}{\small\bfseries\thepage}{}  %
\makeevenhead{chap}{}{}{}   %
\makeoddhead{chap}{}{}{}    %
% \makeheadrule{chap}{\textwidth}{\normalrulethickness}
\makeatother

\nouppercaseheads
\pagestyle{standard}               % Choosing pagestyle and chapter pagestyle
\aliaspagestyle{chapter}{chap} %

% table of contents

\maxtocdepth{subsection} % Only parts, chapters and sections in the table of contents
\settocdepth{subsection}

\AtEndDocument{\addtocontents{toc}{\par}} % Add a \par to the end of the TOC

% new commands
\fi

\newcommand{\ttt}[1]{\texttt{\detokenize{#1}}}
%\newcommand{\todo}[1]{{{\color{purple} TODO: #1}}}


\renewcommand{\subsubsection}[1]{\paragraph{#1.}}

\newcommand{\concat}{\mathbin{||}}

% TODO: move to correct place
\newcommand{\ppt}{\mathsf{PPT}}
\newcommand{\Sign}{\mathsf{Sign}}
\newcommand{\Ver}{\mathsf{Ver}}
\newcommand{\MAC}{\mathsf{MAC}}
\newcommand{\MACSign}{\mathsf{MAC.Sign}}
\newcommand{\MACVerify}{\mathsf{MAC.Verify}}
\newcommand{\INDCPA}{\text{IND-CPA}}
\newcommand{\xor}{\oplus}
\newcommand{\iv}{\text{IV}}
\newcommand{\AES}{\mathsf{AES}}
\newcommand{\ind}{\cong}

%fields and groups
\newcommand{\id}{\mathsf{id}}
\newcommand{\overflow}{\mathsf{overflow}}
\newcommand{\F}{\mathbb{F}}
\newcommand{\Fp}{\F_p}
\newcommand{\Fq}{\F_q}
\newcommand{\Ftwo}{\F_2}
\newcommand{\Q}{\mathbb{Q}}
\newcommand{\N}{\mathbb{N}}
\newcommand{\Z}{\mathbb{Z}}
\newcommand{\R}{\mathbb{R}}
\newcommand{\C}{\mathbb{C}}
\newcommand{\T}{\mathbb{T}}
\newcommand{\Qbar}{\overline{\Q}}
\newcommand{\G}{\mathbb{G}}
\newcommand{\Vs}{\mathbb{V}}
\newcommand{\Fbar}{\overline{\mathbb{F}}}
\newcommand{\Hash}{\mathsf{Hash}}
\newcommand{\Omtilde}{\widetilde{\Omega}}

\newcommand{\Enc}{\mathsf{Enc}}
\newcommand{\Dec}{\mathsf{Dec}}
\newcommand{\Apply}{\mathsf{Apply}}
\newcommand{\Preproc}{\mathsf{Preproc}}


\DeclareMathOperator*{\E}{\textrm{E}}
\DeclareMathOperator*{\argmax}{arg\,max}
\DeclareMathOperator*{\argmin}{arg\,min}


%vectors, etc.
\newcommand{\av}{\mathbf{a}} \newcommand{\cv}{\mathbf{c}}
\newcommand{\dv}{\mathbf{d}} \newcommand{\ev}{\mathbf{e}}
\newcommand{\rv}{\mathbf{r}} \newcommand{\sv}{\mathbf{s}}
\newcommand{\tv}{\mathbf{t}} \newcommand{\uv}{\mathbf{u}}
%\newcommand{\vv}{\mathbf{v}} \newcommand{\wv}{\mathbf{w}}
\newcommand{\xv}{\mathbf{x}} \newcommand{\yv}{\mathbf{y}}
\newcommand{\zv}{\mathbf{z}} \newcommand{\zerov}{\mathbf{0}}
\renewcommand{\AA}{\mathbf{A}} \newcommand{\BB}{\mathbf{B}}
\newcommand{\CC}{\mathbf{C}} \newcommand{\FF}{\mathbf{F}}
\newcommand{\MM}{\mathbf{M}} \newcommand{\RR}{\mathbf{R}}
\renewcommand{\SS}{\mathbf{S}} \newcommand{\TT}{\mathbf{T}}
\newcommand{\UU}{\mathbf{U}} \newcommand{\XX}{\mathbf{X}}
\newcommand{\YY}{\mathbf{Y}} \newcommand{\KK}{\mathbf{K}}
\newcommand{\A}{\mathcal{A}} \newcommand{\B}{\mathcal{B}}
\newcommand{\dash}{\mbox{---}}
\renewcommand{\O}{\mathcal{O}}
\newcommand{\qq}{\mathfrak{q}}
\newcommand{\QQ}{\mathfrak{Q}}
\newcommand{\ZQ}{\Z_{q}}
\newcommand{\ZZ}{\Z}

\newcommand{\vu}{\mathbf{u}}

\newcommand{\todo}[1]{\noindent {\color{blue} {\footnotesize {\bf TODO}:~{#1}}}}
\newcommand{\hcg}[1]{\todo{HCG: #1}}

\newcommand{\mr}[1]{\ensuremath{\mathrm{{#1}}}}
\newcommand{\la}{\ensuremath{\leftarrow}}
\newcommand{\ra}{\ensuremath{\rightarrow}}
\newcommand{\ala}{\ensuremath{\ \la\ }}
\newcommand{\ara}{\ensuremath{\ \ra\ }}
\newcommand{\rf}{\ensuremath{\overset{\$}{\la}}}

\newcommand{\calA}{\ensuremath{\mathcal{A}}}
\newcommand{\calB}{\ensuremath{\mathcal{B}}}
\newcommand{\calC}{\ensuremath{\mathcal{C}}}
\newcommand{\calD}{\ensuremath{\mathcal{D}}}
\newcommand{\calE}{\ensuremath{\mathcal{E}}}
\newcommand{\calF}{\ensuremath{\mathcal{F}}}
\newcommand{\calG}{\ensuremath{\mathcal{G}}}
\newcommand{\calH}{\ensuremath{\mathcal{H}}}
\newcommand{\calI}{\ensuremath{\mathcal{I}}}
\newcommand{\calJ}{\ensuremath{\mathcal{J}}}
\newcommand{\calK}{\ensuremath{\mathcal{K}}}
\newcommand{\calL}{\ensuremath{\mathcal{L}}}
\newcommand{\calM}{\ensuremath{\mathcal{M}}}
\newcommand{\calN}{\ensuremath{\mathcal{N}}}
\newcommand{\calO}{\ensuremath{\mathcal{O}}}
\newcommand{\calP}{\ensuremath{\mathcal{P}}}
\newcommand{\calQ}{\ensuremath{\mathcal{Q}}}
\newcommand{\calR}{\ensuremath{\mathcal{R}}}
\newcommand{\calS}{\ensuremath{\mathcal{S}}}
\newcommand{\calT}{\ensuremath{\mathcal{T}}}
\newcommand{\calU}{\ensuremath{\mathcal{U}}}
\newcommand{\calV}{\ensuremath{\mathcal{V}}}
\newcommand{\calW}{\ensuremath{\mathcal{W}}}
\newcommand{\calX}{\ensuremath{\mathcal{X}}}
\newcommand{\calY}{\ensuremath{\mathcal{Y}}}
\newcommand{\calZ}{\ensuremath{\mathcal{Z}}}

% -- bold math symbols, for some reason --
\newcommand{\boldalpha}{\ensuremath{\boldsymbol{\alpha}}}
\newcommand{\boldchi}{\ensuremath{\boldsymbol{\chi}}}
\newcommand{\boldtau}{\ensuremath{{\boldsymbol{\tau}}}}
\newcommand{\boldstar}{\ensuremath{\mathbf{*}}}
\newcommand{\bolda}{\ensuremath{\mathbf{a}}}
\newcommand{\boldb}{\ensuremath{\mathbf{b}}}
\newcommand{\boldc}{\ensuremath{\mathbf{c}}}
\newcommand{\boldd}{\ensuremath{\mathbf{d}}}
\newcommand{\bolde}{\ensuremath{\mathbf{e}}}
\newcommand{\boldf}{\ensuremath{\mathbf{f}}}
\newcommand{\boldg}{\ensuremath{\mathbf{g}}}
\newcommand{\boldh}{\ensuremath{\mathbf{h}}}
\newcommand{\boldi}{\ensuremath{\mathbf{i}}}
\newcommand{\boldj}{\ensuremath{\mathbf{j}}}
\newcommand{\boldk}{\ensuremath{\mathbf{k}}}
\newcommand{\boldl}{\ensuremath{\mathbf{l}}}
\newcommand{\boldm}{\ensuremath{\mathbf{m}}}
\newcommand{\boldn}{\ensuremath{\mathbf{n}}}
\newcommand{\boldo}{\ensuremath{\mathbf{o}}}
\newcommand{\boldp}{\ensuremath{\mathbf{p}}}
\newcommand{\boldq}{\ensuremath{\mathbf{q}}}
\newcommand{\boldr}{\ensuremath{\mathbf{r}}}
\newcommand{\bolds}{\ensuremath{\mathbf{s}}}
\newcommand{\boldt}{\ensuremath{\mathbf{t}}}
\newcommand{\boldu}{\ensuremath{\mathbf{u}}}
\newcommand{\boldv}{\ensuremath{\mathbf{v}}}
\newcommand{\boldw}{\ensuremath{\mathbf{w}}}
\newcommand{\boldx}{{\ensuremath{\mathbf{x}}}}
\newcommand{\boldy}{\ensuremath{\mathbf{y}}}
\newcommand{\boldz}{\ensuremath{\mathbf{z}}}
\newcommand{\boldzero}{\ensuremath{\boldsymbol{0}}}
\newcommand{\boldone}{\ensuremath{\boldsymbol{1}}}
\newcommand{\boldpi}{\ensuremath{\boldsymbol{\pi}}}
\newcommand{\boldmu}{\ensuremath{\boldsymbol{\mu}}}
\newcommand{\boldPi}{\ensuremath{\boldsymbol{\Pi}}}

% -- bold italic math symbols, for some reason --
\newcommand{\boldia}{\ensuremath{\boldsymbol{a}}}
\newcommand{\boldib}{\ensuremath{\boldsymbol{b}}}
\newcommand{\boldic}{\ensuremath{\boldsymbol{c}}}
\newcommand{\boldid}{\ensuremath{\boldsymbol{d}}}
\newcommand{\boldie}{\ensuremath{\boldsymbol{e}}}
\newcommand{\boldif}{\ensuremath{\boldsymbol{f}}}
\newcommand{\boldig}{\ensuremath{\boldsymbol{g}}}
\newcommand{\boldih}{\ensuremath{\boldsymbol{h}}}
\newcommand{\boldii}{\ensuremath{\boldsymbol{i}}}
\newcommand{\boldij}{\ensuremath{\boldsymbol{j}}}
\newcommand{\boldik}{\ensuremath{\boldsymbol{k}}}
\newcommand{\boldil}{\ensuremath{\boldsymbol{l}}}
\newcommand{\boldim}{\ensuremath{\boldsymbol{m}}}
\newcommand{\boldin}{\ensuremath{\boldsymbol{n}}}
\newcommand{\boldio}{\ensuremath{\boldsymbol{o}}}
\newcommand{\boldip}{\ensuremath{\boldsymbol{p}}}
\newcommand{\boldiq}{\ensuremath{\boldsymbol{q}}}
\newcommand{\boldir}{\ensuremath{\boldsymbol{r}}}
\newcommand{\boldis}{\ensuremath{\boldsymbol{s}}}
\newcommand{\boldit}{\ensuremath{\boldsymbol{t}}}
\newcommand{\boldiu}{\ensuremath{\boldsymbol{u}}}
\newcommand{\boldiv}{\ensuremath{\boldsymbol{v}}}
\newcommand{\boldiw}{\ensuremath{\boldsymbol{w}}}
\newcommand{\boldix}{\ensuremath{\boldsymbol{x}}}
\newcommand{\boldiy}{\ensuremath{\boldsymbol{y}}}
\newcommand{\boldiz}{\ensuremath{\boldsymbol{z}}}

\newcommand{\transpose}[1]{\ensuremath{{#1}^{\intercal}}}

\newcommand{\boldA}{\ensuremath{\mathbf{A}}}
\newcommand{\boldB}{\ensuremath{\mathbf{B}}}
\newcommand{\boldC}{\ensuremath{\mathbf{C}}}
\newcommand{\boldD}{\ensuremath{\mathbf{D}}}
\newcommand{\boldE}{\ensuremath{\mathbf{E}}}
\newcommand{\boldF}{\ensuremath{\mathbf{F}}}
\newcommand{\boldG}{\ensuremath{\mathbf{G}}}
\newcommand{\boldH}{\ensuremath{\mathbf{H}}}
\newcommand{\boldI}{\ensuremath{\mathbf{I}}}
\newcommand{\boldJ}{\ensuremath{\mathbf{J}}}
\newcommand{\boldK}{\ensuremath{\mathbf{K}}}
\newcommand{\boldL}{\ensuremath{\mathbf{L}}}
\newcommand{\boldM}{\ensuremath{\mathbf{M}}}
\newcommand{\boldN}{\ensuremath{\mathbf{N}}}
\newcommand{\boldO}{\ensuremath{\mathbf{O}}}
\newcommand{\boldP}{\ensuremath{\mathbf{P}}}
\newcommand{\boldQ}{\ensuremath{\mathbf{Q}}}
\newcommand{\boldR}{\ensuremath{\mathbf{R}}}
\newcommand{\boldS}{\ensuremath{\mathbf{S}}}
\newcommand{\boldT}{\ensuremath{\mathbf{T}}}
\newcommand{\boldU}{\ensuremath{\mathbf{U}}}
\newcommand{\boldV}{\ensuremath{\mathbf{V}}}
\newcommand{\boldW}{\ensuremath{\mathbf{W}}}
\newcommand{\boldX}{\ensuremath{\mathbf{X}}}
\newcommand{\boldY}{\ensuremath{\mathbf{Y}}}
\newcommand{\boldZ}{\ensuremath{\mathbf{Z}}}

\newcommand{\deq}{\mathrel{\mathop:}=}
\newcommand{\zo}{\ensuremath{\{0,1\}}} % bits
\newcommand{\bin}{\zo}
\newcommand{\zon}{\ensuremath{\{0,1\}^n}} % bits

% Theorem definitions

\theoremstyle{plain}
%\newtheorem{theorem}{Theorem}[section]
\newtheorem{theorem}{Theorem}[section]
\newtheorem{inf-theorem}{Informal Theorem}[section]
\newtheorem*{rtheorem}{Theorem}
\newtheorem{lemma}[theorem]{Lemma}
\newtheorem{corollary}[theorem]{Corollary}
\newtheorem*{claim}{Claim}
\theoremstyle{remark}
\newtheorem{remark}[theorem]{Remark}
\theoremstyle{definition}
\newtheorem*{theorems}{Theorem}
\newtheorem{defn}[theorem]{Definition}
\newtheorem{definition}[theorem]{Definition}
\newtheorem{fact}[theorem]{Fact}
\newtheorem{conjecture}[theorem]{Conjecture}
\newtheorem{const}[theorem]{Construction}
\newtheorem{attackgame}[theorem]{Attack Game}


\newtheoremstyle{goal}% name of the style to be used
  {\topsep}% measure of space to leave above the theorem. E.g.: 3pt
  {\topsep}% measure of space to leave below the theorem. E.g.: 3pt
  {\normalfont}% name of font to use in the body of the theorem
  {0pt}% measure of space to indent
  {\bfseries}% name of head font
  {: } %punctuation between head and body
  { }% space after theorem head; " " = normal interword space
  {\thmname{#1}\thmnumber{ #2}\thmnote{ (#3)}}
\theoremstyle{goal}
\newtheorem{sgoal}{Security Goal}
\newtheorem{fgoal}{Functionality Goal}

\newtheoremstyle{pstyle}% name of the style to be used
  {2pt}% measure of space to leave above the theorem. E.g.: 3pt
  {2pt}% measure of space to leave below the theorem. E.g.: 3pt
  {\itshape}% name of font to use in the body of the theorem
  {0pt}% measure of space to indent
  {\itshape}% name of head font
  {: } %punctuation between head and body
  { }% space after theorem head; " " = normal interword space
  {\thmname{#1}\thmnumber{ #2}\thmnote{ (#3)}}
\theoremstyle{pstyle}
\newtheorem{prop}[theorem]{Proposition}
%% custom macros

\newcommand{\esm}[1]{\ensuremath{#1}}
\newcommand{\ms}[1]{\esm{\mathsf{#1}}}

\newcommand{\prg}{\ms{PRG}}
\newcommand{\prf}{\ms{PRF}}
\newcommand{\prp}{\ms{PRP}}

\newcommand{\poly}{\operatorname{poly}}
\newcommand{\polylog}{\operatorname{polylog}}
\newcommand{\negl}{\operatorname{negl}}

\newcommand{\ord}[1]{\esm{{#1}^{\mr{th}}}}

\newcommand{\hyb}{\ms{Hyb}}
\newcommand{\Funs}{\ms{Funs}}
\newcommand{\getsr}{\rgets}
\newcommand{\rgets}{\mathrel{\mathpalette\rgetscmd\relax}}
\newcommand{\rgetscmd}{\ooalign{$\leftarrow$\cr
    \hidewidth\raisebox{1.2\height}{\scalebox{0.5}{\ \rm R}}\hidewidth\cr}}

%\getsr with proper vertical space.    
% this makes it typeset better in subscripts
\def\getsrx{\mathrel{%
    \mathchoice{\GETSRX}{\GETSRX}{\scriptsize\GETSRX}{\tiny\GETSRX}%
}}
\def\GETSRX{{%
\setbox0\hbox{$\gets$}%
\rlap{\hbox to \wd0{\hss$\raisebox{1.2\height}{\scalebox{0.5}{\ R}}$\hss}}\box0
}}

\newcommand{\Unif}{\ms{Unif}}

\newcommand{\iseq}{\stackrel{?}{=}}

\newcommand{\appc}{\stackrel{c}{\approx}}
\newcommand{\apps}{\stackrel{s}{\approx}}
\newcommand{\abs}[1]{\left| #1 \right|}

\newcommand{\classP}{\ms{P}}
\newcommand{\classBPP}{\ms{BPP}}
\newcommand{\classNP}{\ms{NP}}
\newcommand{\Ppoly}{\ms{P}/\ms{poly}}
\newcommand{\NCone}{\ms{NC^1}}

\newcommand{\round}[1]{\left\lfloor #1 \right\rceil}
\newcommand{\floor}[1]{\left\lfloor #1 \right\rfloor}
\newcommand{\ceil}[1]{\left\lceil #1 \right\rceil}
\newcommand{\iprod}[1]{\left\langle #1 \right\rangle}
\newcommand{\norm}[1]{\left\| #1 \right\|}
\newcommand{\set}[1]{\left\{ #1 \right\}}

\newcommand{\Zp}{\Z_p}
\newcommand{\Zq}{\Z_q}

\newcommand{\subind}[2]{{#1}^{(#2)}}

\newcommand{\eqq}{\stackrel{?}{=}}

\newcommand{\Otilde}{\widetilde{O}}
\newcommand{\Otildel}{\widetilde{O}_\lambda}
\newcommand{\Omtildel}{\widetilde{\Omega}_\lambda}

\newcommand{\hc}{\ms{hc}}

\newcommand{\xstar}{x^*}
\newcommand{\ystar}{y^*}

\newcommand{\DDHAdv}{\ms{DDHAdv}}
\newcommand{\PRFAdv}{\ms{PRFAdv}}
\newcommand{\PRGAdv}{\ms{PRGAdv}}

\newcommand{\ct}{\ms{ct}}

\newcommand{\View}{\ms{View}}
%\newcommand{\view}{\ms{View}_{\calV^*}}
\newcommand{\Sim}{\ms{Sim}}


\newcommand{\bv}{\mathbf{b}}
\newcommand{\Bm}{\mathbf{B}}
\newcommand{\Rm}{\mathbf{R}}
\newcommand{\Gm}{\mathbf{G}}

\newcommand{\pro}[1]{\langle #1 \rangle}
\newcommand{\Comm}{\ms{Commit}}
\newcommand{\cSAT}{\ms{circuit}\text{-}\ms{SAT}}

\newcommand{\op}{\ms{op}}
\newcommand{\ck}{\ms{ck}}
\newcommand{\st}{\ms{st}}
\newcommand{\shint}{\hint_s}
\newcommand{\chint}{\hint_c}
\newcommand{\rhint}{\hint_r}
\newcommand{\fhint}{\hint_f}
\newcommand{\row}{\ms{row}}
\newcommand{\col}{\ms{col}}

\newcommand{\sig}{\sigma}

\newcommand{\sk}{{\ms{sk}}}
\newcommand{\pk}{\ms{pk}}
\newcommand{\pp}{\ms{pp}}
\newcommand{\vk}{\ms{vk}}


\newcommand{\seed}{\ms{seed}}
\newcommand{\db}{\ms{db}}
\newcommand{\qu}{\ms{qu}}
\newcommand{\irow}{i_{\text{row}}}
\newcommand{\icol}{i_{\text{col}}}
\newcommand{\adv}{\ms{adv}}
\newcommand{\off}{\ms{off}}

\newcommand{\hreq}{q_h}
\newcommand{\hint}{\mathsf{hint}}

\newcommand{\ans}{\mathsf{ans}}
\newcommand{\Query}{\mathsf{Query}}
\newcommand{\Decide}{\mathsf{Decide}}
\newcommand{\Setup}{\mathsf{Setup}}
\newcommand{\Hint}{\mathsf{Hint}}
\newcommand{\SAns}{\mathsf{Answer}}
\newcommand{\Ans}{\SAns}
\newcommand{\Answer}{\SAns}

\newcommand{\defeq}{\coloneqq}
\newcommand{\distadv}{\mathsf{DistAdv}}
\newcommand{\piradv}{\mathsf{PIRadv}}
\newcommand{\epspir}{\epsilon_\mathsf{pir}}
\newcommand{\epsbox}{\epsilon_\mathsf{box}}
\newcommand{\Supp}{\mathsf{Supp}}
\newcommand{\Bern}{\mathsf{Bernoulli}}


\newcommand{\PS}{\Psi}
\mathchardef\mhyphen="2D % Define a "math hyphen"
\newcommand{\pPRF}{\mathsf{pPRF}}
\newcommand{\psk}{\sk_\mathsf{p}}
\newcommand{\pS}{S_\mathsf{p}}
\newcommand{\pb}{b_\mathsf{p}}
\newcommand{\Kp}{{\mathcal{K}_{\mathsf{p}}}}
\newcommand{\noti}{{\bar{i}}}


\newcommand{\Gen}{\mathsf{Gen}}
\newcommand{\Punc}{\mathsf{Punc}}
\newcommand{\Eval}{\mathsf{Eval}}
\newcommand{\InSet}{\mathsf{InSet}}
\newcommand{\Choose}{\mathsf{Choose}}
\newcommand{\GenWith}{\mathsf{GenWith}}
\newcommand{\Shift}{\mathsf{Shift}}
\newcommand{\Derive}{\mathsf{Derive}}

\newcommand{\PRFGen}{\mathsf{PRFGen}}
\newcommand{\PRFPunc}{\mathsf{PRFPunc}}
\newcommand{\PRFEval}{\mathsf{PRFEval}}

\newcommand{\PSadv}{\mathsf{PSAdv}}
\newcommand{\PSseladv}{\PS\mathsf{selAdv}}
\newcommand{\pPRFadv}{\mathsf{pPRFAdv}}
\newcommand{\pPRFadvu}{{\mathsf{pPRFAdv}^\mathrm{unp}}}
\newcommand{\prstuple}{(\Gen,\Punc,\Eval)}
\newcommand{\PStuple}{(\Gen,\allowbreak\Punc,\allowbreak\Eval)}
\newcommand{\PRFtuple}{(\PRFGen,\allowbreak\PRFPunc,\allowbreak\PRFEval)}
\newcommand{\Pituple}{\Pi =\allowbreak (\Setup,\allowbreak\Hint,\allowbreak\Query,\allowbreak\SAns,\allowbreak\CRecon)}
\newcommand{\PreTuple}{\PiPre =\allowbreak (\Setup,\allowbreak\Query,\allowbreak\Hint,\allowbreak\Ans,\allowbreak\Recon)}

\newcommand{\getsrtiny}{\mathrel{\mathpalette{
    \ooalign{\text{\tiny{$\leftarrow$}}\cr%
    \hidewidth\raisebox{1\height}{\scalebox{0.7}{\,\rm{\tiny R}}}\hidewidth\cr}
}\relax}}

\newcommand{\PrTiny}[1]{\Pr_{\scriptscriptstyle{#1}}} 

\newcommand{\ia}{{i_{\mathsf{adv}}}}
\newcommand{\ip}{i_{\mathsf{punc}}}
\newcommand{\ig}{i_{\mathsf{guess}}}
\newcommand{\ipir}{i_{\mathsf{pir}}}
\newcommand{\dummy}{\mathsf{{dummy}}}
\newcommand{\new}{\mathsf{{new}}}
\newcommand{\sknew}{\sk_{\ms{new}}}
\newcommand{\skright}{\sk_{\ms{right}}}
\newcommand{\skleft}{\sk_{\ms{left}}}
\newcommand{\Prbig}[1]{\Pr\Big[ #1 \Big]}

\newcommand{\ol}{1^{\lambda}}
\newcommand{\fk}{\ms{k}}
\newcommand{\fkp}{\ms{k}_{\ms{p}}}


\makeatletter
\def\moverlay{\mathpalette\mov@rlay}
\def\mov@rlay#1#2{\leavevmode\vtop{%
   \baselineskip\z@skip \lineskiplimit-\maxdimen
   \ialign{\hfil$\m@th#1##$\hfil\cr#2\crcr}}}
\newcommand{\charfusion}[3][\mathord]{
    #1{\ifx#1\mathop\vphantom{#2}\fi
        \mathpalette\mov@rlay{#2\cr#3}
      }
    \ifx#1\mathop\expandafter\displaylimits\fi}
\makeatother

% Vectors and Matrices
\renewcommand{\vec}[1]{\mathbf{#1}}
\newcommand{\mat}[1]{\ensuremath{\mathbf{#1}}\xspace} % Matrix

\newcommand{\Accept}{\ensuremath{\mathsf{Accept}}\xspace} 
% Algorithms
\newcommand{\GenToken}{\ensuremath{\mathsf{GenToken}}\xspace}
\newcommand{\KDer}{\ensuremath{\mathsf{KDer}}\xspace} % Key Derivation
\newcommand{\KeyComp}{\ensuremath{\mathsf{KeyComp}}\xspace} % Key Compression
\newcommand{\NextStep}{\ensuremath{\mathsf{NextStep}}\xspace}
\newcommand{\OAccess}{\ensuremath{\mathsf{OAccess}}\xspace} % Oblivious Access
\newcommand{\PoK}[2]{\ensuremath{\mathsf{PoK}\left\{#1 : #2\right\}}\xspace}
\newcommand{\ReRand}{\ensuremath{\mathsf{ReRand}}\xspace} % Re-randomize
\newcommand{\Read}{\ensuremath{\mathsf{Read}}\xspace}
\newcommand{\Write}{\ensuremath{\mathsf{Write}}\xspace}
\newcommand{\access}{\ensuremath{\mathsf{Access}}\xspace} % Access
\newcommand{\aggregate}{\ensuremath{\mathsf{Agg}}\xspace} % Aggregate
\newcommand{\auth}{\ensuremath{\mathsf{Auth}}\xspace}
\newcommand{\com}{\ensuremath{\mathsf{Com}}\xspace}
\newcommand{\constrain}{\ensuremath{\mathsf{Constrain}}\xspace} % Constrain
\newcommand{\corr}{\ensuremath{\mathsf{Corr}}\xspace}
\newcommand{\delegate}{\ensuremath{\mathsf{Del}}\xspace}
\newcommand{\enroll}{\ensuremath{\mathtt{enrl}}\xspace}
\newcommand{\evict}{\ensuremath{\mathsf{Evict}}\xspace} % Evict
\newcommand{\fetch}{\ensuremath{\mathsf{Fetch}}\xspace} % Fetch
\newcommand{\finalize}{\ensuremath{\mathsf{Finalize}}\xspace}
\newcommand{\foward}{\ensuremath{\mathsf{Fwd}}\xspace}
\newcommand{\gen}{\ensuremath{\mathsf{Gen}}\xspace}
\newcommand{\join}{\ensuremath{\mathsf{Join}}\xspace}
\newcommand{\judge}{\ensuremath{\mathsf{Judge}}\xspace} % Judge
\newcommand{\link}{\ensuremath{\mathsf{Link}}\xspace}
\newcommand{\open}{\ensuremath{\mathsf{Open}}\xspace}
\newcommand{\prove}{\ensuremath{\mathsf{Prove}}\xspace}
\newcommand{\rand}{\ensuremath{\mathsf{Rand}}\xspace}
\newcommand{\register}{\ensuremath{\mathsf{Reg}}\xspace}
\newcommand{\request}{\ensuremath{\mathsf{Request}}\xspace}
\newcommand{\preprocess}{\ensuremath{\mathsf{Preprocess}}\xspace}
\newcommand{\respond}{\ensuremath{\mathsf{Respond}}\xspace}
\newcommand{\rotate}{\ensuremath{\mathsf{Rotate}}\xspace}
\newcommand{\setup}{\ensuremath{\mathsf{Setup}}\xspace}
\newcommand{\encrypt}{\ensuremath{\mathsf{Encrypt}}\xspace}
\newcommand{\decide}{\ensuremath{\mathsf{Decide}}\xspace}
\newcommand{\store}{\ensuremath{\mathsf{store}}\xspace}
\newcommand{\test}{\ensuremath{\mathsf{Test}}\xspace}
\newcommand{\trace}{\ensuremath{\mathsf{Trace}}\xspace}
\newcommand{\update}{\ensuremath{\mathsf{Update}}\xspace}
\newcommand{\validate}{\ensuremath{\mathtt{val}}\xspace}
\newcommand{\query}{\ensuremath{\mathsf{Query}}\xspace}
\newcommand{\answer}{\ensuremath{\mathsf{Answer}}\xspace}
\newcommand{\Index}{\ensuremath{\mathsf{Index}}\xspace}

\newcommand{\Conv}{\mathsf{Conv}}
\newcommand{\Share}{\mathsf{Share}}
\newcommand{\recover}{\mathsf{Recover}}
\newcommand{\Decode}{\mathsf{Dec}}
\newcommand{\expand}{\mathsf{Decomp}}
\newcommand{\expandinv}{\mathsf{Recomp}}
\newcommand{\Combine}{\mathsf{Combine}}
\newcommand{\Findcover}{\mathsf{FindCover}}
\newcommand{\Tofuncs}{\mathsf{ToFuncs}}

\newcommand{\tofunc}{\mathsf{Expand}}
\newcommand{\Bad}{\mathsf{Bad}}
\newcommand{\cross}{\times}


