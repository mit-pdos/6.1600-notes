In the last section, our strategy for
authentication depended on two parties sharing a
secret key.
In that discussion, we completely left
out of the picture how these parties should
exchange this secret key.
Our implication was that they
went to some private room and exchanged the key in
secret, but in many cases this is not practical:
if they could whisper a key, why not just whisper the message?

Luckily, there is a way to get around this requirement for a shared secret using \emph{public-key cryptography}.\cite{DH76} % TODO: cite DH
\marginnote{The original Diffie-Hellman paper from 1976, which introduced
public-key cryptography, is a fascinating read.}

\section{Definitions}
The basic idea of public-key cryptography, applied
to authentication, is that each party will
generate two linked keys---a secret signing key
and a public verification key.
The verification key will be good enough to verify that a signature
is valid, but not to generate new signatures.

\begin{definition}[Signature Scheme]
	A signature scheme is associated with a message space $\calM$ and three efficient algorithms $(\Gen, \Sign, \Ver)$.

      \marginnote{In theoretical papers, people will write $\Gen(1^\lambda)$ to indicate that the key-generation
      algorithm takes as input a length-$\lambda$ string of ones.
      This is just a hack to make the input given to $\Gen$ $\lambda$ bits long so that the
      $\Gen$ algorithm can run in time polynomial in its input length: $\poly(\lambda)$.
      If we express $\lambda$ in binary, then $\Gen(\lambda)$ gets a $\log_2 \lambda$-bit input
      and can only run in time $\poly(\log \lambda)$.
      This distinction is really unimportant, but if you see the $1^\lambda$ notation, you will
      now know what it means.}
	\begin{itemize}
    \item $\Gen(\lambda) \to (\sk, \vk)$.
      The key-generation algorithm as input a security parameter $\lambda \in \N$ and outputs a secret signing key $\sk$ and public verification key $\vk$.
      The algorithm $\Gen$ runs in time $\poly(\lambda)$.
    \item $\Sign(\sk, m) \to \sigma$.
      The signing algorithm takes as input a secret key $\sk$ and a message $m \in \calM$, and outputs a signature $\sigma$.
    \item $\Ver(\vk, m, \sigma) \to \zo$.
      The signature-verification algorithm takes as input a public verification key $\vk$, a message $m \in \calM$, and a signature $\sig$, 
      and outputs $\bin$, indicating acceptance or rejection.
	\end{itemize}
	
\end{definition}

For a signature scheme to be useful, a correct verifier must always accept messages from an
honest signer. Formally, we have:

\begin{definition}[Digital signatures: Correctness]
  A digital-signature scheme $(\Gen, \Sign, \Ver)$ is \emph{correct} if,
  for all messages $m \in \calM$:
  \[ \Pr\big[\Ver(\vk, m, \Sign(\sk, m)) = 1 \colon (\sk, \vk) \gets \Gen(\lambda) \big] = 1. \]
\end{definition}

The standard security notion for digital signatures is very similar
to that for MACs (\cref{def:mac-sec}).
The only difference here is that a digital-signature scheme splits the single
secret MAC key into two keys: a secret signing key and a public verification key.
Otherwise the definition is essentially identical.

\begin{definition}[Digital signatures: Security -- existential unforgeability under chosen message attack]\label{def:sig-sec}
  A digital-signature scheme $(\Gen, \Sign, \Ver)$ is \emph{secure} if
  all efficient adversaries win the following security 
  game with only negligible probability:
  \begin{itemize}[noitemsep]
    \item The challenger runs $(\sk, \vk) \gets \Gen(\lambda)$ and sends $\vk$ to the adversary.
    \item For $i = 1, 2, \dots$  (polynomially many times)
      \begin{itemize}
        \item The adversary sends a message $m_i \in \calM$ to the challenger.
        \item The challenger replies with $\sigma_i \gets \Sign(\sk, m_i)$.
      \end{itemize}
    \item The adversary outputs a message-signature pair $(m^*, \sigma^*)$.
    \item The adversary wins if $\Ver(\vk, m^*, \sigma^*) = 1$ and $m^* \not \in \{m_1, m_2, \dots\}$.
  \end{itemize}
\end{definition}

Notice that the security definition here allows an attacker, given a valid
message-signature pair $(m, \sigma)$ to produce additional valid message-signature
pairs on the same message: $(m, \sigma'), (m, \sigma''), \dots$.
Standard digital-signature schemes, such as the elliptic-curve digital signature
algorithm (EC-DSA) have this property.

In some applications, we want to prohibit an attacker from finding \emph{any}
new message-signature pair. We call this security notion ``\emph{strong} existential unforgeability under chosen message attack.'' 

The definition is the same as in \cref{def:sig-sec} except that we require
the adversary to find a valid-message signature pair $(m^*, \sigma^*)$
such that $(m^*, \sigma^*) \not \in \{ (m_1, \sigma_1), (m_2, \sigma_2), \dots \}$.

\section{Constructing a Signature Scheme}
In the following sections, we will show how to construct a digital-signature
scheme from any one-way function (\cref{def:owf}).

We will generate a signature scheme that is secure, but that has a relatively large
signatures and public keys: to achieve security against attackers running
in time $2^\lambda$, this signature scheme has signatures of $O(\lambda^2)$ bits.
Widely used modern digital signature schemes (e.g., EC-DSA) have signatures
of $O(\lambda)$ bits.\marginnote{One benefit of the signature scheme that 
we present here is that---unlike EC-DSA, RSA, DSA, and other widely used
signature schemes---this one is plausibly secure even against \emph{quantum}
adversaries.\todo{Cite NIST PQ signature schemes and compare}
}

We will construct this scheme in three stages:

\begin{enumerate}
	\item Construct a \emph{one-time secure} signature scheme for \emph{fixed-length messages}.
        With this scheme, an attacker who sees two signatures under the same signing key can forge signatures.
        In addition, the secret signing key for this scheme will be larger than the size of the message 
        being signed.
      \item Construct a \emph{one-time secure} scheme for \emph{arbitrary-length messages}.
        Here, we construct a one-time signature scheme whose secret signing key is independent of 
        the length of the signed message.

      \item Construct a \emph{many-time secure} scheme (i.e., a fully secure one under 
        \cref{def:sig-sec}) for \emph{arbitrary-length messages}.
        This last scheme is a fully secure and fully functional digital-signature scheme.
\end{enumerate}


\section{One-time-secure Signatures (Lamport Signatures)}

In this section we give a very simple and elegant construction 
of one-time-secure digital signatures, due to Lamport.\cite{L79}
Before giving the construction, we define one-time security for 
digital-signature schemes.
This signature scheme is not generally useful on its own, but is
useful as a building block.

\begin{definition}[Digital signatures: One-Time Security]\label{def:sig-once}
A digital-signature scheme $(\Gen, \Sign, \Ver)$ over message space $\calM$ is \emph{one-time secure} if all efficient adversaries win
the following game with negligible probability:
  \begin{itemize}[noitemsep]
    \item The challenger generates $(\sk, \vk) \leftarrow \Gen(\lambda)$ and sends $\vk$ to the adversary.
    \item The adversary sends the challenger \emph{single} message $m \in \calM$.
    \item The challenger responds with $\sig = \Sign(\sk, m)$.
    \item The adversary outputs $(m^*, \sig^*)$.
    \item The adversary wins the game if $\Ver(\vk, m^*, \sig^*) = 1$ and $m^* \neq m$.
\end{itemize}
\end{definition}

\paragraph{Lamport signatures.}
We now construct a one-time secure signature scheme for messages in $\bin^n$,
for some fixed message length $n \in \N$. 
To do this, we will define the following algorithms, which make use of a
one-way function $f \colon \calX \to \calY$:
\begin{itemize}
  \item $\Gen() \to (\sk, \vk)$. 
    \marginnote{In this construction, we leave the security parameter $\lambda$
    implicit.
    To be fully formal, $\Gen$ would take $\lambda$ an input.
    The one-way function $f$ and its domain $\calX$ would both
    depend on $\lambda$. So we would write $f_\lambda$ and $\calX_\lambda$.
    }
    Choose $2n$ random elements from $\calX$,
    the domain of the one-way function $f$.
    Arrange these values in to a $2 \times n$
    matrix, which forms the secret signing key $\sk$.
    The public verification key just consists of the $2n$
    images of these values under the one-way function $f$:
\[ \sk \gets \begin{pmatrix}
	x_{10} & \ldots & x_{n0} \\
	x_{11} & \ldots & x_{n1} \\
	\end{pmatrix},\quad \vk \gets \begin{pmatrix}
	f(x_{10}) & \ldots & f(x_{n0}) \\
	f(x_{11}) & \ldots & f(x_{n1}) \\
\end{pmatrix}.\]
	\item $\Sign(\sk, m) \to \sigma$ outputs $(x_{1m_1}, \ldots x_{nm_n})$, where $m_1 \dots m_n$ 
    are the individual bits of the length-$n$ message $m \in \zo^n$.
	\item $\Ver(\vk, m, \sig) \to \zo$ parses the 
    the message into bits $m = m_1\dots m_n \in \zon$ and
    the signature $\sig$ into its individual symbols $\sig = (x_1^*, \ldots x_n^*) \in \calX^n$.
    The signing routine accepts if, for all $i \in \{1, \dots, n\}$:
    \begin{equation}
      f(x^*_i) = \vk_{i,m_i}.\label{eq:lamport}
    \end{equation}
    In other words, the routine accepts if applying the one-way function $f$ to each symbol
    of the signature matches the corresponding value in the verification key.
    (Otherwise, the signing routine rejects.)
\end{itemize}

This signature scheme has relatively large keys:
the verification key, in particular consists of $2n$ values,
where each is of length $\Omega(\lambda)$ bits.
So the total length is roughly $2n\lambda$ bits---much
longer than the $n$-bit message being signed.

In addition, notice that an adversary who sees signatures
on even two messages can forge signatures on messages of its choice.
In particular:
\begin{itemize}[noitemsep]
  \item The adversary first asks for a signature on the message $m_0 = 0^n$.
It receives $\sig_0 = (x_{10}, \ldots, x_{n0})$.
  \item The adversary then then asks for a signature on the message $m_1 = 1^n$.
    It receives $\sig_1 = (x_{11}, \ldots, x_{n1})$.
  \item At this point, the adversary has the entire secret signing key! 
\end{itemize}
However, we will show that this scheme is indeed one-time secure.

\begin{claim} 
The Lamport signature scheme is one-time secure under the
assumption that $f$ is a one-way function.
\end{claim}
\marginnote{Remember that if $\classP = \classNP$, 
one-way functions, and also digital signature schemes, do not exist. 
So any proof of security of a digital-signature scheme will require
some sort of cryptographic assumption.}

In cryptography, we generally prove these security
claims by \emph{reduction}: we will show that
if there exists an efficient adversary $\calA$
that breaks the security of our scheme,
then we can construct an efficient adversary $\calB$ 
that breaks one of our assumptions.
If we do this, we have reached a contradiction to one
of our assumptions, so the first adversary cannot exist.

\begin{proof}[Proof of Claim]
Suppose there exists an adversary $\A$ that wins the 
one-time-security game of \cref{def:sig-once} with non-negligible probability $\epsilon$.
That is, the adversary can produces $(m^*, \sig^*)$ such that $\Ver(\vk, m^*, \sig^*) = 1$ and $m \neq m^*$ given only $\sig = \Sign(\sk, m)$.
We can then construct an adversary $\B$ that can use $\A$ to 
invert the one-way function.

\marginnote{Lamport's construction shows that if one-way functions
exist, then so do digital signatures.
Can you show that if digital signatures exist, then so do one-way
functions?}

In particular, our adversary $\calB$ will use algorithm $\calA$
as a subroutine to invert the one-way function.
We will show that if $\calA$ wins in the one-time signature security
game often, then algorithm $\calB$ will invert the one-way function
often, which is a contradiction.

Assume our one-way function is of the form $f \colon \calX \to \calY$
and that the Lamport signature scheme is on $n$-bit messages.
The one-way-function adversary $\calB$ operates as follows:
\begin{itemize}[noitemsep]
  \item The adversary $\calB$ is given a point $y \in \calY$
    and its task is to produce a preimage of $y$ under $f$. 
  \item The adversary $\calB$ generates a signing keypair as follows:
    \begin{itemize}[noitemsep]
      \item It runs the key-generation algorithm for the Lamport signature scheme $(\sk, \vk) \gets \Gen()$.
      \item The adversary chooses a random value $i^* \getsr \{1, \dots, n\}$
            and a random bit $\beta^* \getsr \zo$.
          \item The adversary sets $\vk_{i^*,\beta^*} \gets y$.
            That is, it inserts the one-way-function point it must invert
            into a random location in the verification key.
    \end{itemize}
  \item The adversary then sends the verification key $\vk$ to the Lamport-signature adversary $\calA$.
  \item The adversary $\calA$ asks for the signature on a message $m = m_1 m_2 \dots m_n \in \zon$.
  \item If $m_{i^*} = \beta^*$, then algorithm $\calB$ cannot produce a valid signature on the message
        $m$ and it outputs FAIL.
  \item Otherwise, the algorithm $\calB$ returns the signature $\sigma = (\sk_{1,m_1}, \dots, \sk_{n,m_n}) \in \calX^n$
        to algorithm $\calA$.
  \item Algorithm $\calA$ then produces a forged message-signature pair $(m^*, \sigma^*)$,
        where $m \neq m^*$.
  \item Algorithm $\calB$ parses $m^* = m^*_1 \dots m^*_n \in \zon$
        and $\sigma^* = \sigma^*_1 \dots \sigma^*_n \in \calX^n$. Then:
        \begin{itemize}[noitemsep]
          \item If $m_{i^*} = m_i$, algorithm $\calB$ outputs FAIL.
          \item Otherwise, algorithm $\calB$ outputs $x \gets \sigma^*_{i^*} \in \calX$.
        \end{itemize}
\end{itemize}

First, notice that whenever $(m^*, \sigma^*)$ is a valid message-signature
pair and whenever algorithm $\calB$ does not output FAIL, algorithm $\calB$
outputs a preimage $x \in \calX$ of point $y \in \calY$ under the one-way function $f$.
That is because, by the verification relation (\ref{eq:lamport}) for Lamport signatures,
\[f(x) = f(\sigma^*_{i^*}) = \vk_{i^*,m^*_{i^*}} = \vk_{i^*, 1 - m_i} = \vk_{i^*, \beta^*} = y.\]

Now, we must show that algorithm $\calB$ does not output FAIL too often.
Since algorithm $\calB$ chooses the values $i^*$ and $\beta^*$ at random,
and since the adversary $\calA$ behavior is \emph{independent} of these values,
we can say:
  \begin{itemize}
    \item the probability of the first failure event is $1/2$,
          since there are two possible choices of $m_{i^*}$ and only 
          one of these is bad, and 
    \item the probability of the second failure event is at most $1/n$,
          since $m$ and $m^*$ must differ in at least one of $n$ bits,
          and there is a $1/n$ probability that this differing bit is
          at index $i^*$.
  \end{itemize}

The events that $\calA$ breaks the signature scheme
and that either of these failures occur are all \emph{independent}.
Then if $\calA$ breaks the one-way function with probability $\epsilon$,
our one-way-function adversary $\calB$
inverts the one-way function with probability
\[ \epsilon_\text{one-way} = \epsilon \cdot \frac{1}{2} \cdot \frac{1}{n}.\]

The probability of either bad is at most $1/2 + 1/n$,
by the union bound.
Therefore if algorithm $\calA$ breaks one-time security of Lamport's
scheme with probability $\epsilon$,
If $\epsilon$ is non-negligible, then $\epsilon_\text{one-way} = \epsilon/2n$
is also non-negligible, and we have a contradiction.
\end{proof}
