\chapter{Encryption in Practice}

So far, we have established several constructions
that allow us to hide the contents of
transmissions: we created chosen-plaintext- 
and chosen-ciphertext-secure
encryption schemes that worked with and without
a shared key.
We will now discuss a few practical applications
of transport encryption, and why it is often difficult
to get right.


\section{File Encryption}
Perhaps the most straightforward use of encryption is file encryption.

\paragraph{Example: WhatsApp Encrypted Backup.}
WhatsApp allows the app's users to back up their 
messages and contacts to the cloud.
This way, a user can recover her messages 
if she loses or breaks her phone.
To hide the user's data from WhatsApp's cloud servers, WhatsApp
uses encrypted backup.
To achieve this, the user's device generates
a 128-bit AES key $k$ at the time of backup
and encrypt the message data
(photos, messages, etc.) using AES-GCM$(k, \cdot)$
before sending the \emph{ciphertext} to the
server. In order to allow you to restore your
backup on a new phone, the app allows you to
export 64 decimal digits that encode the AES key
used. When restoring your backup, you will enter
these digits and your phone will fetch the
ciphertext from the server and use these digits to
reconstruct the key and decrypt your messages.
\marginnote{See \href{https://scontent.fphl1-1.fna.fbcdn.net/v/t39.8562-6/241394876_546674233234181_8907137889500301879_n.pdf?_nc_cat=108&ccb=1-7&_nc_sid=ad8a9d&_nc_ohc=ANW9FNYPigIAX93N8Oj&_nc_ht=scontent.fphl1-1.fna&oh=00_AT9IVosU4uWqddd5woEO4eUWw3mbd76IgvwjcqDApb3p9A&oe=63565D66}{\textbf{this document}}
for details on how WhatsApp encrypts backups.
(The document also describes a more complicated
backup scheme that uses passwords for encryption.)}

This is a fairly simple application of file security. However, file encryption can be much more tricky: many applications require or desire features beyond simple encryption and decryption.

\subsection{Case Study: PDF v1.5 Encryption}
One instance of this desire for extra features
that ended up going wrong was a previous version
of the PDF standard, PDF v1.5.\autocite{muller2019practical}
This standard provided several features:
\begin{compactenum}
	\item It is possible to password-encrypt some or all of the document.
        (The encryption scheme does not matter, but think of it as
        a secure authenticated-encryption scheme.)
        For example, a PDF could have an unencrypted title page 
        and have the rest of the pages be encrypted.
	\item A PDF document can contain a form that the PDF reader
        submits to a server via HTTP.
	\item A form in a PDF document can reference other parts of a document.
	\item The PDF reader may submit a form when an event happens:
        when the PDF is opened, closed, decrypted, etc.
\end{compactenum}

Each of these features individually seems innocuous.
However, when combined, there is a clever attack that allows an attacker to 
learn the contents of the encrypted portion of a PDF.

The attack works against a PDF document with an \emph{unecrypted (public)
title page}
and an \emph{encrypted (secret) body}.
We assume that the attacker can modify the PDF 
file on its way to the victim.

To mount the attacker, the attacker intercepts the
PDF on the way to the victim and replaces the
title page with an \textquote{evil} title page.
The evil title page:
\begin{itemize}[noitemsep]
    \item contains an invisible PDF form,
    \item reads the contents of the decrypted pages into a PDF form element,
          on the event that decryption of the PDF body succeeds, 
          and then
    \item submits the form via HTTP to \texttt{evil.com}
\end{itemize}
So when the victim recipient enters her password into the PDF
reader to decrypt the document, the evil title page is exfiltrates
the decrypted content to \texttt{evil.com} via a PDF form.

\paragraph{What went wrong?}
The core issue here was that the unencrypted
contents of the document were not authenticated.
That is, an attacker could modify the
unencrypted pages of the document without detection.
A better design would have been to either use
a MAC over all pages of the document or to use
a primitive called ``Authenticated Encryption with
Associated Data'' to authenticate the encrypted data
together with the unencrypted data.

One downside of MACing the entire document is that,
before rendering even a single page of the PDF,
the reader would have to compute a MAC over the
entire PDF.
If the PDF consists of many thousands of pages, 
this could be expensive.
Using a more sophisticated cryptographic construction:
per-page MACs, plus MACs on the document metadata, etc.,
we might be able to construct a scheme that protects
document integrity while allowing partial document
rendering.
But the design of such a scheme would have to carefully
defend document integrity against the sort of attacks
we describe here.

\section{Stream Encryption: Transport Layer Security (TLS)}
The standard Internet data transfer protocols---TCP and UDP---%
provide no integrity or confidentiality whatsoever.
To protect the data that we send over the network, 
we use the \emph{transport-layer security} (TLS) protocol.
The TLS protocol runs on top of TCP and aims to 
provide an encrypted \textquote{tunnel} between a
client and server.\marginnote{HTTPS is simply HTTP run over
TLS.}

While designing a stream-encryption protocol may seem straightforward
at first glance, the task is much more subtle than it would seem.
However, as is often the case in security, 
features and practical requirements make the situation much more complex.

\subsection{Downgrade Attack}
We now give one example of a \emph{downgrade attack}---the
sort of subtle security issue that can arise in protocol design.

The current version of TLS is TLS 1.3. An older
version of the TLS standard, called SSLv3,
is vulnerable to devastating
attacks that allow an attacker to recover the plaintext traffic.
However, for backwards compatibility, many TLS clients
and servers still supported SSLv3 until quite recently.

When a TLS client connects to a TLS server, the two parties
first exchange some messages to decide which version of the TLS
protocol to use.
In particular, the client will try to connect to the server using
the latest version of TLS it supports.
The server will respond back with either a confirmation 
(if the server supports the client's proposed TLS version) or with garbage
(if not).
If garbage, the client will try to connect to the server
with an older version of TLS.

An important point is that \emph{none} of these negotiation messages
are authenticated---the client and server cannot start
using authentication (MACs, signatures, etc.) until they agree on 
which version of TLS to use!
So, an active in-network attacker
can simply replace all of the server responses in
this protocol-negotiation phase
with garbage until the client downgrades all the way to SSLv3.
Once the client and server agree to use SSLv3,
believing that this is their best available option,
the attacker can then monitor and decrypt their
traffic using known attacks on SSLv3.

The best defense against this attack is for both parties to
disable support for SSLv3 completely.

\subsection{TLS Structure}
TLS consists of two main phases:
\begin{compactenum}
\item \textbf{Handshake:}
  In this phase, the client and server use a key-exchange protocol to agree on a shared key to use to encrypt their application-layer traffic. This step uses public-key cryptography, since the client and server initially have no shared secret.
\item \textbf{Record protocol:} This phase is where the actual application-layer communication happens. This phases uses the secret key that the client and server agreed  upon in the handshake phase for authentication and encryption. 
\end{compactenum}

\subsection{TLS Handshake Properties}
In our definitions of public-key encryption, we
had only two (relatively simple) properties: 
correctness and security. 
The TLS handshake, however, has a much more complicated
set of goals:

\begin{compactitem}
\item \textbf{Correctness:} Both parties agree on the same session key 
        at the conclusion of the handshake.
      \item \textbf{Security:} adversary \textquote{learns nothing} about the secret
        key that the parties agree upon at the conclusion of the handshake.
      \item \textbf{Peer authentication:} At the end of the handshake, each party believes that they are talking to the other party.
      \item \textbf{Downgrade protection:} The parties agree on the same version of TLS that they would agree on absent an in-network attacker.
      \item \textbf{Forward secrecy with respect
        to key compromise:} if an attacker
        steals the secrets stored on the client or the server, 
        the attacker cannot encrypt past traffic.
        \marginnote{To provide forward secrecy, modern cryptographic
        protocols use long-term secrets only for signing---not for
        encryption.
        These protocols use \emph{ephemeral} (one-time-use) 
        cryptographic keys for key exchange and encryption.
        Protocol participants delete these ephemeral keys on
        connection teardown and/or they rotate these keys often.

        This way, if an attacker compromises a party's secret key,
        the attacker can only sign messages on behalf of that party;
        the attacker cannot use the secret to decrypt past messages.}
      \item \textbf{Protection against key-compromise impersonation:} If an attacker steals a client's secret key, the attacker should not be able to impersonate other servers to the client.
      \item \textbf{Protection of endpoint
        identities:} The public keys of the two
        parties should never flow over the wire in
        the clear. For example, if a client is
        connecting to a website that uses
        a content-distribution network, such as
        Akamai or Cloudflare, an attacker should
        not be able to tell which website the
        client is connecting to---only that it is
        hosted on Akamai or Cloudflare.
\end{compactitem}

\subsection{TLS Handshake}
The TLS handshake is very carefully designed to achieve these properties. A grossly simplified version looks something like the following:

\begin{compactenum}
\item At the start of the handshake, the TLS client
  holds the public $\pk_{\text{CA}}$ of a certificate authority.
  The TLS server (for example, MIT) holds its secret signing key
  $\sk_\text{MIT}$ and a public-key certificate $\text{cert}_{MIT}$
  binding its public key $\pk_\text{MIT}$ to its domain \texttt{mit.edu}.
\item \textbf{Client Hello}: The client sends the following values to
  the server:
	\begin{compactitem}
		\item a random nonce,
		\item list of supported ciphersuites, and
      \marginnote{A \emph{ciphersuite} contains all of the
          cryptographic parameters needed to perform key exchange,
          hashing, authentication, and encryption.
          For example, one possible ciphersuite for TLS 1.2 is
          \texttt{ECDHE-RSA-AES256-GCM-SHA384}.
          This indicates use of ephemeral elliptic-curve
          Diffie-Hellman key exchange (\texttt{DHE}) with
          RSA signatures (\texttt{RSA}), 256-bit AES
          encryption in GCM mode (\texttt{AES256-GCM}),
          and SHA2-384 as the hash function (\texttt{SHA384}).}
		\item an ephemeral Diffie-Hellman public key.
          (The client constructs this Diffie-Hellman public key
          using its preferred ciphersuite. If the server does not
          support the ciphersuite the client picked, the client
          will have to re-run this step using a different
          ciphersuite.)
	\end{compactitem}
\item \textbf{Server Hello}: The server sends several values to the client, choosing a ciphersuite to use and completing the Diffie-Hellman key exchange. 
  That is, the server sends:
	\begin{compactitem}
		\item a random nonce,
		\item a ciphersuite to use,
		\item and an ephemeral Diffie-Hellman public key
	\end{compactitem}
\item Both partices compute a shared session key $k$
  using Diffie-Hellman key agreement on the ephemeral keys 
  they exchanged.
\item Under encryption using the keys derived from the session key $k$, the server sends the certificate for \ttt{mit.edu} as well as a signature over all messages sent so far, using its long-term secret key $\sk_\text{MIT}$. The client then checks that:
	\begin{compactitem}
	\item the certificate has been signed by one of the client's trusted CAs, and
	\item the signature from the server matches their own record of the messages.
	\end{compactitem}
\item Finally, the client and server run the TLS Record Protocol to exchange encrypted and authenticated application data.
\end{compactenum}

This simplified toy version of the TLS handshake 
does not provide many of the features that the real TLS handshake
provides. But it should give you a flavor of what the real 
handshake looks like.

% TODO: protection of endpoint identities

\section{Properties that TLS does not provide}

\paragraph{Authenticated End-of-File}
TLS does not provide any end-of-file authentication,
or ``clean closure.''
To explain what this means by example:

A popular tool to install the toolchain for the trendy systems programming language Rust is \ttt{rustup}. To use the tool and install the Rust toolchain, the recommanded method is to run the 
command ``\ttt{curl https://sh.rustup.rs | sh}.''
This downloads a shell script from the internet 
over HTTPS and immediately runs it using the shell \ttt{sh}.

Imagine that the contents of the downloaded script
create a temporary directory, copy things into it,
install some things, and finally delete the
temporary directory with something like \ttt{rm -r
/tmp/install}.
An in-network attacker, who knows the structure of the \ttt{rustup}
install script, could drop all of the packets in the stream immediately
after the characters \ttt{rm -r /}.
Eventually, the TLS connection will timeout, an shell will run
the command \ttt{rm -r /}, deleting the user's entire file system.

To protect against this, script writers try to
design their scripts such that if the stream is
cut off in the middle of a download, nothing happens.
For example, the install script might consist of a 
single function definition that is called at the very 
end of the function.

\paragraph{Plaintext Length Obfuscation}
As we have discussed, encryption reveals exactly
the length of the plaintext. If there is data that
is not encrypted that is then included inside the
encrypted data as well, this can cause
a vulnerability---see the CRIME attack.
% TODO: Finish this up. It's a clever attack.
