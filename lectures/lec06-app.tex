\chapter{Factoring integers}
\label{sec:fact}

The problem of integer factorization was central to 20th-century cryptography.
Breaking the one-wayness of the RSA trapdoor one-way function (\cref{chap:rsa}), for example,
is no harder than factoring integers.
In this chapter, we will see a couple of surprisingly powerful algorithms for factoring integers.

We will only consider factoring numbers of the form $N=pq$, for distinct odd primes $p$ and $q$.
(The general case is not too much more challenging.)
Throughout, let $n = \lceil \log_2 N \rceil$ be the bitlength of the number to factor.

\section{Background}
\label{sec:fact:bg}

\paragraph{Trial division.}
We can factor $N$ by trying to divide $N$ by each of the primes of size $\leq \sqrt N$
and checking whether the result is an integer.
If so, we have found a factor of $N$.
Since at least one of the two factors of $N$ is in $\{1, \dots, \sqrt N\}$, this
algorithm (``trial division'') runs in time roughly $\sqrt{N} = 2^{n/2}$.

Trial division is an \emph{exponential time} algorithm, since it runs in time $2^{\Omega(n)}$,
where the bitlength $n$ is the size of the number to be factored.
The best known factoring algorithms run in \emph{sub-exponential time} $2^{O(n^c)}$,
for some constant $c < 1$.


\paragraph{Euclid's algorithm.}
An important subroutine in almost all factoring algorithms is Euclid's polynomial-time algorithm
for computing the greatest common divisor of two integers $x$ and $y$.\marginnote{In this
discussion, we draw on Arjen Lenstra's very nice survey on factoring~\cite{lenstra2000integer}.}

The principle of Euclid's algorithm is that
\[ \gcd(x, y) = \gcd(x, y \bmod x) \qquad \text{and} \qquad \gcd(x, 0) = x.\]
So, for example, if we want to compute $\gcd(46, 12)$, we can compute it as:
\[ \gcd(46,12) = \gcd(12, 10) = \gcd(10, 2) = 2. \]

\paragraph{Difference of squares.}
The second key idea is that, if we can find two numbers $x,y \in \Z$ whose
squares are congruent modulo $N$, we can use these numbers to factor $N$:
\begin{align*}
x^2 &= y^2 &\pmod N \\
x^2 - y^2 &= 0 &\pmod N\\
(x+y)(x-y) &= 0 &\pmod N\\
\end{align*}

If $x = \pm y$, then this relation is not helpful to us.\marginnote{It is
not necessarily obvious that the useful pairs $(x,y)$ will ever exist.
The key idea is that, modulo $N=pq$, ever number in $\Z^*_N$ either has
four square roots or has none. If an element in $\Z^*_N$
 has four square roots then the roots
are of the form $r,-r, s, -s$. In this case, a pair $(\pm r, \pm s)$ yields the
sort of relation that we need to factor.}
But if $x \neq \pm y$, then we know that
$x+y \neq 0 \bmod N$ and $x-y \neq 0 \bmod N$.
So we have:
\[ (x+y)(x-y) = kN \quad \in \Z,\]
for some positive integer $k \in \Z$.
Then $x+y$ must be a multiple of one of the factors of $N$ (but not both),
and $\gcd(x+y, N)$ reveals a factor of $N$. 

The goal of many factoring algorithms---including the one we will see today---%
is finding these integers $x$ and $y$ whose squares are congruent modulo $N$.

\section{Dixon's algorithm}

Dixon's algorithm is one of the simplest sub-exponential-time factoring algorithms.
It gives a fast method for finding two numbers whose squares are congruent modulo $N$.
Once we have these squares, we can use them to factor as in \cref{sec:fact:bg}

\subsection{The idea}

The principle of Dixon's algorithm is that we will pick 
many random numbers $r \in \Z^*_N$ and square them modulo 
the integer $N$ we would like to factor.

Say that we are somehow able to find numbers $r, r'$ such that
\begin{align*}
  r^2 &= 2 \cdot 3^2 \cdot 5 &\pmod N\\
  r'^2 &= 2 \cdot 5 &\pmod N,\\
\intertext{then we know that:}
(rr')^2 &= 2^2 \cdot 3^2 \cdot 5^2 &\pmod N\\
(rr')^2 &= (2 \cdot 3 \cdot 5)^2 &\pmod N
\end{align*}
and now we have two numbers whose squares
are congruent modulo $N$:
\[ x = r r' \qquad \text{ and }\qquad y = 2 \cdot 3 \cdot 5.\]
If we are lucky, this is the useful type of congruence that
we can use to factor $N$ (i.e., $r r' \neq \pm 2 \cdot 2 \cdot 5 \bmod N$).

The principle of Dixon's algorithm is to generate many such $r$s
and then use linear algebra to find a subset of them whose product
modulo $N$ is a perfect square. 

\subsection{The algorithm}

\paragraph{Input:} An integer $N = pq$ for odd primes $p$ and $q$. A parameter $B \in \N$,
which we refer to as ``the size of the factor base.''

\paragraph{Output:} The factors $(p,q)$ of $N$.


\begin{enumerate}
\item \textbf{Collect linear relations.}
      Maintain a list $L$ of pairs of 
      (a) an element in $\Z^*_N$ and (b) vectors over $\Z_2^B$.
      Repeat until $L$ contains $B+1$ pairs:
      \begin{itemize}
          \item Sample $r \getsr \Z^*_N$.
          \item Compute $s \gets (r^2 \bmod N)$.
          \item Attempt to write $s$ as a product of the first $B$ primes:
            \[ s = 2^{e_2} 3^{e_3} 5^{e_5} \dots \]\marginnote{If
            a number completely splits into prime factors $\leq B$,
          we say that the number is ``$B$-smooth.''}
          \item If successful, add the pair $(r, (e_2, e_3, e_5, \dots))$
                to the list~$L$.

          
      \end{itemize}

\item \textbf{Solve linear system.}
        Let $L = \{ (r_1, \vec v_1), (r_2, \vec v_2), \dots \}$.
        Find a non-zero combination of the vectors in $L$
        that sums to zero modulo $2$.
        That is, find $S \subseteq [B+1]$ such that
\[ \sum_{i \in S} \vec v_i = (0, 0, 0, \cdots, 0) \qquad \in \Z_2^B. \]
    \marginnote{We can find the set $S$ using Gaussian elimination
    in roughly $B^3$ time.
    Since the set of vectors will be extremely sparse, there
    are faster methods that implementers use in practice.}
Letting 
\[ (e_2, e_3, e_5, \dots) \gets \sum_{i \in S} \vec v_i,\]
        we then have a difference of squares:
    \[ \left(\sum_{i \in S} r_i \right)^2 = (2^{(\frac{e_2}{2})} \cdot 3^{(\frac{e_3}{3})} \cdot 5^{(\frac{e_5}{2})} \cdots )^2 \qquad \pmod N.\]

  \item \textbf{Use Euclid's algorithm to try to factor $N$.}
    We can take:
\begin{align*}
  x &= \left(\sum_{i \in S} r_i \right)\\
  y&= 2^{(\frac{e_2}{2})} \cdot 3^{(\frac{e_3}{3})} \cdot 5^{(\frac{e_5}{2})} \cdots )
\end{align*}
     and compute $\gcd(x+y, N)$.
     With probability roughly $1/2$, over the random choice of the $r$s, this will
     yield a factor of $N$.
\end{enumerate}

\subsection{The analysis.}

The costs of the three steps of the algorithms are:
\begin{enumerate}
  \item Each iteration of the loop requires us to try to 
        factor a number into primes $\leq B$.
        We can factor in this way by trial division 
        using time roughly $B$.\marginnote{I'm ignoring any $\log B$ factors,
        which do matter very much in practice.}
        
        The question then is how many trials it will
        take for us to find a single smooth number.
        For a smoothness bound $B$, let's say for
        now that it takes $T(B)$ trials---we will look into 
        the precise value of $T(B)$ in a moment..

  \item Solving the linear system using Gaussian elimination 
    takes roughly $B^3$ time.
  \item Run Euclid's algorithm---the time required here is 
        negligible compared to the time of the first two steps.
        This step runs in time $\poly(\log N) = \poly(n)$.
\end{enumerate}

Putting everything together, we have that 
factoring an $n$-bit number with a factor base of
size $B$ takes time: 
\begin{align}
B \cdot T(B) + B^3 + \poly(n). \label{eq:time}
\end{align}


\paragraph{Smoothness probabilities.}
The key question that we need to answer to complete
the analysis is
\begin{quote}
``If we pick an integer uniformly at random from $\{1, \dots, N\}$,
  what is the probability that the integer will be $B$-smooth?''
\end{quote}

The convention is to denote the number of $B$-smooth 
numbers in $\{1, \dots, N\}$ as $\Psi(N,B)$.
When $B$ is ``not too small,'' we have:\marginnote{For many
more details on these estimates, take a look at Granville's
very nice survey on smooth numbers.~\cite{granville2008smooth}
}
\[ \Psi(N, B) \approx N \cdot u^{-u + o(1)}\qquad \text{for } u = \frac{\log N}{\log B}.\]

The probability of a random number modulo $N$ being $B$-smooth is
then $\Psi(N,B)/N$ and the expected number of tries it will
take for us to find a smooth number is:
\[ T(B) = 1/\Psi(N,B).\]

Now we can plug this estimate into the expression (\ref{eq:time})
for the running time of Dixon's algorithm and we can solve
for the value of $B$ that minimizes the running time.
In particular, to minimize the running time we want:
\begin{align*}
  B \approx T(B) \approx N / \Psi(N,B) = u^u,
\end{align*}
for $u = (\log N)/(\log B)$.

\begin{align*}
  B &= u^u\\
  \log B &= u \log u\\
  \log B &= \frac{\log N}{\log B} \log \frac{\log N}{\log B}\\
  \log^2 B &\approx \log N \log \log N\\
\log B &\approx \sqrt{\log N \log \log N}\\
B &\approx \exp(\sqrt{\log N \log \log N}).\\
\end{align*}

If we plug this value of $B$ into Dixon's algorithm,
we get a running time of 
\[ \exp(O(\sqrt{ \log N \log \log N}) = 2^{O(\sqrt{ n \log n })}.\]



