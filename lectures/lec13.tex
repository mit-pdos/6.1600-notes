\chapter{Open Questions in Encryption}

So far, we have established what may seem like
a comprehensive set of tools to transmit data over
the network: we have schemes for verifying the
integrity of data and for hiding the contents of
a transmission from an adversary, both with and
without a shared key.

Transport-layer security (TLS) effectively builds
and \textquote{encrypted pipe} between a client and a server.
Through the encrypted pipe that TLS provides, we can run 
any of our favorite TCP-based
protocols---HTTP, SMTP, POP, IMAP, etc.---and can
thus hide our data from an in-network attacker.
And yet, the security guarantees that TLS provides fall
short of the strongest possible security notions we could
desire.

If we were to imagine the best possible security
we could ask for regarding network traffic, it
might look (imprecisely) something like the
following:

\begin{framed}\noindent
\emph{An attacker who controls many parties (clients and servers) as well as the network should \textquote{learn nothing} about who the client is talking to and what she is saying.}
\end{framed}

Unfortunately, the protocols we have for secure communication today
fall far short of this goal.
In this section, we describe some of the shortcomings of today's
transport-security tools and some imperfect solutions.

\section{Problem: Encryption does not hide\\the source and destination of a packet}

\iffalse
In order to even begin talking to
a server, the client must learn the server's IP
address. This happens using the DNS protocol,
which is entirely unencrypted---a network
adversary can easily watch the DNS queries a client
makes and learn who it is talking to.
\fi

In order to send IP packets over the
internet, the Internet's routing system relies
on routers in the network knowing the source
and destination IP addresses of each packet: these are
included, unencrypted, in the packet header. 
(In some ways, the \textquote{pipe} analogy fits
here: anyone can see where the pipe starts and
where it ends.)

\paragraph{Solution Attempt: Tor}
The Tor system aims to allow a client to connect
to a server over the Internet while hiding---from certain
types of adversaries---which server the client is connecting
to.
For example, a Tor client should be able to browse the web
without anyone learning which websites the client is visiting.

Tor's strategy is to 
bounce traffic around the Internet and hope
that no real-world attacker can gather enough information
to figure out which client is communicate with which server.
Tor provides no precise security guarantees, and there are
scores of research papers demonstrating various weaknesses
in Tor's security plan.
At the same time, Tor is publicly available,
is well supported, is widely used, and 
seems to provide some meaningful privacy benefits in practice.%
\marginnote{It is difficult to evaluate the security of a tool
like Tor, since real-world attackers will not necessarily reveal
that they can break the tool's security guarantees. So using
a tool like Tor requires taking a leap of faith.}


Tor works by nesting several of these encrypted pipes: when opening a connection,
the Tor client will first select three \emph{relays} from the Tor network $(A, B, C)$. 
The Tor client will then:
\begin{enumerate}
  \item Open an encrypted tunnel to the first relay $A$ (the ``guard'').
	\item Through that tunnel, open an encrypted tunnel to the second relay~$B$.
	\item Through that tunnel-inside-a-tunnel, open an encrypted tunnel to the
    third relay $C$ (the ``exit'').
\end{enumerate}

The client will then send its application-layer traffic 
through this tunnel-inside-a-tunnel-inside-a-tunnel.
So each byte of application data will be encrypted first for relay
$C$, then for relay $B$, then for relay $A$.
When the client sends this ciphertext over the
\emph{circuit} from relay $A$ to $B$ to $C$ to the
real destination, relay $A$ will first strip off its
layer of encryption then forward the inner packet
to relay $B$. Relay $B$ will do the same, stripping off
a layer of encryption and forwarding the packet to
relay $C$. Finally, relay $C$ will strip off the last layer of
encryption and be left with a normal IP packet
that it can then send to the destination server.
As the response makes it back through the network,
each relay node will add a layer of encryption.
The end result of this is that no single relay
can see the source and destination IP addresses.

However, the security that Tor provides is imperfect.
First, if an attacker controls the guard node (relay $A$) and the
exit node (relay $C$), the attacker can correlate the timing of
when a packet enters the guard node and when a packet
exits the exit node.
Using this timing an attacker can make a guess at the route
traffic is taking through the Tor network.
Even without controlling relay nodes, if an attacker 
controls certain key points in the Internet (e.g., Internet exchange points
or undersea fiber links) it may be able to perform this sort
of traffic analysis even without controlling relays.

\section{Problem: Attacker sees packet sizes and timings}
As we discussed, practical encryption
schemes necessarily reveal the length of the
ciphertext. In the context of the Internet,
this means that an attacker can learn the size
of each TCP packet that a client sends, along
with timing information.
(Here the ``encrypted pipe'' analogy for TLS
breaks down: an attacker can see how much traffic
flows through the encrypted pipe and when.)

Even without seeing the destination and source of packets sent to and from 
a client's machine, an attacker can learn significant amounts about 
the client's traffic.
Some examples are:
\begin{itemize}
  \item \emph{Watching a movie:} Video traffic has a distinct traffic pattern.
      By monitoring, for example, the length of time that a client spends 
      watching a movie, a network attacker learn with 
      fairly high accuracy \emph{which movie} the client is watching.
  \item \emph{Using \ttt{ssh}:} 
      Different commands will have different traffic patterns.
      An attacker may be able to infer what type of commands a client 
      is running by inspecting traffic patterns.
  \item \emph{Downloading a file:}
      The bitlength of a downloaded file can uniquely identify the
      file in many cases.
  \item \emph{Browsing the web:} sizes leak individual pages.
\end{itemize}

\paragraph{Example: New York Times} The New York
Times homepage \ttt{nytimes.com/} downloads 1.56
MB of content, along with 76 total assets (images, CSS,
JavaScript). The webpage to submit
a sensitive tip, \ttt{nytimes.com/tips}, downloads
41.92 KB and only 15 assets. By counting the
number of HTTPS requests that a client makes over an
encrypted connection to \ttt{nytimes.com}, an attacker 
can easily distinguish whether a client is
visiting the homepage or the tips page, even if
the attacker cannot decrypt even a single bit of
the HTTPS traffic itself.

\paragraph{Attempts at a solution}
There are several common ways that people attempt
to protect against this sort of traffic analysis.
None of these solutions works well. 

\begin{enumerate}
  \item \textbf{Random Noise:} 
    To try to hide the length of the packets it sends,
    a client can add a randomly chosen number of bytes
    of dummy data to the end of each packet.
    The hope is that by randomizing packet lengths, the
    client prevents the attacker from performing the
    traffic analysis.

    Unfortunately, a patient attacker can use \emph{averaging}
    to effectively eliminate the effect of the random noise.
    That is, if the attacker can trick the client into sending
    the same message a few times (as is often possible), the
    attacker can average the noised packet lengths to get a good estimate
    of the true length.

\item \textbf{Padding:}
  Another option is to just pad every packet (or
  webpage or encrypted message, etc.) to match the
  largest packet that the client will ever send.
  For example, whenever the client visits a page on
  \ttt{nytimes.com}, the client could
  download 50MB of page content and 100 fixed-size assets, even
  if the true page is tiny.
  This is somewhat secure, but
  incredibly costly and therefore not practical
  outside of very specific circumstances.
\end{enumerate}

\section{A Promising Direction: Metadata Privacy for Messaging}
Messaging apps like WhatsApp and iMessage are end-to-end encrypted, but still may leak who you are talking to. This problem is more tractable due to the circumstances of messaging:

\begin{itemize}
	\item Messages are approximately fixed length.
	\item Some latency is OK.
	\item Total daily traffic per user is small.
	\item Each user talks to few messaging partners.
\end{itemize}

Because of these constraints, it is somewhat feasible to use techniques like padding to greatly reduce the amount of data that messaging metadata reveals and to do so in a way that provides strong formal guarantees about security.

\section{Compromised Servers}
Even a perfect encrypted pipe that hides metadata, however, is useless if the server gets compromised (or you do not trust the server). The server learns all your data, and is free to lose it in a breach, sell it, or turn it over to law enforcements, etc. We will discuss ways to reduce the impacts of server compromise, but there are also applications of cryptography that can help.

\subsection{Private Information Retrieval}
A common task between client and server is to read a record from a database. In many cases, we might like to be able to do this without the server learning which record we accessed.\marginnote{A concrete application of this is Google search---in order to give you search results, Google necessarily learns what you are searching. With a PIR scheme, it would be possible for Google to look up search results without learning what you are searching for!}

Consider a server database containing $x_1, x_2, \ldots, x_n \in \bin$. For correctness, we would like that an honest client requesting a record $x_i$ from a server gets the correct $x_i$ in response. For security, our goal is that the clients query is a CPA-secure encryption of the index $i$, meaning that the server can learn nothing about $i$. A simple solution would be for the server to simply return all $n$ bits to the user and allow them to choose the correct bit on the client side. For something like Google search, this would obviously be an infeasible amount of data. Surprisingly, this is possible with $\ll n$ bits of communcation.

To achieve this, we need a new tool called \emph{additively homomorphic encryption}. An additively homomorphic encryption scheme is a CPA-secure secret key encryption scheme $(\Enc, \Dec)$ for messages in $\calM = \Z_p$ with the added property:
\[ \forall k \in \calK,\, \forall m, \hat{m} \in \calM,\, \Enc(k, m) \star \Enc(k, \hat{m}) = \Enc(k, m + \hat{m}) \]
In English, additively homomorphic encryption allows us to learn the encryption of the addition of two messages with only the encrypted messages. This also allows us to multiply by constants, and therefore to compute a matrix-vector product of an encrypted vector and a public matrix.

It is possible to construct an additively homomorphic encryption scheme from the DDH assumption, with only a slight tweak to ElGamal encryption.

\subsubsection{PIR Construction}
We can use this to construct a private information retrieval scheme. To do so, the server must store its database (the $x_i$ values) as a $\sqrt{N} \times \sqrt{N}$ matrix $D$. The client then tells the server which column $j$ it would like by supplying the encryption of a $\sqrt{N} \times 1$ vector $\vec{m}$ with a 1 in the $j$th location. The client sends this encryption (using a key only the client knows) as $\Enc(k, \vec{m})$. The server computes the matrix product $\Enc(k, D\vec{m})$, which gives the $j$th column of the matrix, using additively homomorphic encryption and returns the response to the client, where the client can find the bit they are interested in in the column. 

This allows a client to retrieve a bit from a server's database without the server learning anything about the desired bit, and to do so with only $\sqrt{N}$ communication cost. The server computation cost is high---the server necessarily touches every bit of the database---but this is a very promising idea.
