\chapter{Privilege Separation}


The last chapter left off on a somewhat
unsatisfying note: our software is bound to have
bugs and those bugs are likely to be exploitable.
Today, we will try to limit the impact that these
bugs can have by \emph{planning for compromise}
and aiming to limit the damage that each component
can cause if an attacker compromises it.

The philosophy that we will employ is called the \emph{Principle of Least Privilege}. 
The idea is to give each component in the system the least privileges needed to do 
its job.\marginnote{In real systems, it is often costly to implement least privilege in
the strictest sense possible. You should think of the principle of least privileges as
a difficult-to-achieve design goal, rather than a rule that every component of every
system must satisfy.}
By limiting the access that each component of our software has to do sensitive actions, we can prevent the compromise of our whole system when a single piece is attacked.

To summarize our strategy:
\begin{quote}
  \textbf{Principle of Least Privilege:}
	Each component should have the smallest set of privileges necessary to do its job.	
\end{quote}


As a concrete example, a web server might have some networking code, it might
talk to a database server, and it might use some cryptographic keys.
One way to architect the system with least privilege would be to split the
server into multiple distinct processes---one that receives network requests,
one that interfaces with the database, and one that uses the cryptographic keys.
These components then would speak to each other using narrow, restrictive APIs.

In some sense, a system that perfectly implements the principle of least privilege 
gives one with the ``best security we could ask for,'' in the face of component compromise.
\marginnote{An orthogonal, but important, strategy is to reduce the number of privileges
that a component needs to do its job.}
However, most real systems do not completely implement the principle of least privilege
for a number of reasons:

\paragraph{Challenge: Splitting Boundaries.} 
A large piece of software has many components that often have many points of interaction.
To effectively implement least privilege, we need to find
boundaries at which we can separate our software. 
\marginnote{If we don't split up a large piece of software
into components, the software just consists of one \"uber-component.
An attacker who breaks into this one component can hijack the entire system.
}
A few common ways to partition a system or application are:
\begin{itemize}
  \item \textbf{Isolating by user:} In an operating system, different users of the systems
                      have different privileges. This way, if an attacker compromises
                      one user's account it does not compromise the entire system.

\item \textbf{Splitting by feature:} In large business applications, different features
        (e.g., search, user management, mail) run as isolated components.
        This way, a bug in one component does not necessarily affect the entire system.
\item \textbf{Splitting off buggy code:} The Firefox web browser uses special sandboxes to compartmentalizes bug-prone video codec code (\cref{sec:split:codec}).
  This way, if a malicious website is able to exploit the codec code, it is not
    so easy for the attacker to compromise the system.
\item \textbf{Separating exposed versus internal code:} Google's front-end HTTP servers 
    are isolated from the servers that run core application logic.
    This way, an attacker on the network is one step removed from the
    servers with access to Google's internal resources (databases, etc.).

  \item \textbf{Isolating sensitive data or keys:} Certificate authorities (CAs) use hardware security modules to isolate the cryptographic keys from all other code in the system. Laptops and servers sometimes have separate cryptographic co-processors for this purpose as well. 

\end{itemize}

\paragraph{Challenge: Interface/API.}
If we chop our monolithic app into different
domains, we also need to clearly define an API
that the different domains can use to interact. 
This API must allow us to preserve the
functionality of the system while making it difficult
for an attacker that compromises one component from
compromising an adjacent one.

Some strategies for interfacing between different components
of a system might be
\begin{itemize}
\item remote procedure calls (RPCs) over a network,
\item message queues,
\item shared memory,
\item shared database, or
\item shared files or directories.
\end{itemize}

In designing interfaces between isolated components,
we tend to worry about:
\begin{itemize}
  \item \textbf{Functionality:} Does the isolation plan allow the application to work as it should?
  \item \textbf{Security:} Does the isolation strategy prevent the compromise of one component from affecting others?
  \item \textbf{Performance:} How much performance overhead does the isolation mechanism add?
  \item \textbf{Complexity:} How much code does the isolation mechanism require?
If the compartmentalization strategy requires a large
amount of code, this code might introduce new vulnerabilities.
    
\end{itemize}

\section{Example: Logging}
Most systems use logging to keep track of the actions that an application has performed.
That way, if an attacker compromises the application, system administrators can 
look at the log to see what happened and how to mitigate it.
For a logging system to be useful, it must be difficult for an attacker who compromises
the application to erase the log.

It is very natural to separate out the log into a separate component:
the app's functionality is almost entirely
separate from the log, and the app's interface to
the log can be very simple.
The app should be allowed to append to the log and
read from the log---the app should 
\emph{not} be allowed to remove entries from the
log. This way, compromise of the application or log server
will not be enough to erase the log.
\begin{figure}
\begin{verbatim}
Application  ----- Log entries -----> Log server
  server                                  |
     |                                    |   Append/Read API
     |                                    |   (No delete)
     |                                    v
     -------- No access -----> X         Log
\end{verbatim}
\caption{One way to architect a logging system.
  An attacker that compromises the application and/or log
  server may not be able to erase the logs.}
\end{figure}

\section{Example: Cryptography Keys}
Many applications use cryptographic keys for authentication: certificate authorities need to generate signed certificates for their clients, cryptocurrency wallets need to sign transactions, and WebAuthn requires an authenticator to sign a challenge from the relying party. 
In all of these applications, we worry a lot about an attacker stealing the application's secret signing key.
Towards the goal of protecting cryptographic secrets, we often
isolate the code that uses cryptographic keys into a separate software (or even hardware)
component.
In particular, we might create a cryptography component whose 
only API is \ttt{sign(msg)} and \ttt{get_public_key()}.
In this way, even if an attacker compromises the application, it cannot
easily extract the secret key.

This design does not completely protect us against the compromise of the application.
In particular, an attacker who compromises our app can still call \ttt{sign()} as much as they like to sign any message they like. 
At the same time, having the key isolated to this crypto component allows us to add checks inside the crypto component to, for example, reject messages of the wrong type\marginnote{Although in many cases the things we are worried about an attacker asking for signatures of will pass this type check: for example, for a CA, this would still allow an attacker to generate a certificate for their malicious website.}. If our service is a certificate authority, we could set up our crypto component to verify that each message is an actual signature before signing it.

\begin{figure}
\begin{verbatim}
Application  <----- Requests -----> Key manager
     |                                    ^
     |                                    |  Sign API
     |                                    |  (No extraction 
     |                                    |     of key)
     |                                    v
     -------- No access -----> X     Signing key
\end{verbatim}
\caption{A common architecture for isolating cryptographic
keys from an application.}
\end{figure}
This division also allows our crypto code to do some meaningful logging: imagine that the crypto module saves every signature it creates to a log. Compromise of the main app would allow an attacker to generate arbitrary signatures, but administrators could then see every signature that was generated during the compromise. This would be very helpful in recovery.

\section{Example: Media Codecs in Web Browsers}\label{sec:split:codec}
Media codecs (e.g., for JPEG decoding) are notoriously complex and bug-prone:
codec libraries have been at the source of many web-browser exploits. 
If we were able to isolate these codecs, we could
make it less likely that a codec bug allows an attacker to
access data other than the media file being decoded.

\begin{figure}
\begin{verbatim}
Browser tab  ----- Encoded data-----> JPG/Media codec
             <------- Bitmap -------- 
\end{verbatim}
\caption{The Firefox browser isolates media codecs
in a separate sandbox.}\label{fig:codec}
\end{figure}

Isolating codecs may not always be as trivial as \cref{fig:codec} makes it seem.
In many cases, codecs require a sophisticated interface to the browser tab: the
codec library may progressively decode videos as data arrives, for example. 
Even image decoders progressively decode images, improving resolution as more data arrives. Supporting this functionality without adding vulnerabilities or excessive 
latency requires careful API design.

Firefox isolates these risky codecs using
the language-level isolation that the WebAssembly language provides.

\section{Example: Server for Network Time Protocol (NTP)}
Operating systems use the network-time protocol
to fetch the current time from time servers on 
the Internet, and to update the current
time to match. 
Setting the time requires root
privilege on Unix-like systems, but NTP also
requires network accesses; 
the networking code can be complicated
and bug-prone. To prevent an attacker who finds
a bug in these network protocols from gaining root
privilege on our machine, many systems separate
the two into a process that handles talking to the
NTP server on the network and a privileged process
that accepts the time from this other process and
sets it. This way, an attacker who find a bug in
the network code can only set the time---they
cannot perform arbitrary actions as root.
In addition, the time service may impose some
policy on time changes (e.g,. that time can never
go backwards). 

\begin{figure}
\begin{verbatim}
Launcher --------> Network <---> Web
   |               Service                 
   |                  |
   |                  | "adjust clock"
   |                  v
   |-------------> Time Service [Policy enforcement]
\end{verbatim}
\caption{Modern operating systems implement 
a network-time protocol (NTP) client as separate processes.
  This way, an attacker who compromises the client over
  the network cannot arbitrarily corrupt the system time.}\label{fig:ntp}
\end{figure}

\section{Example: OpenSSH Server}
A secure shell (SSH) server has access to many sensitive resources:
network port 22, a host secret key, the system's password file, and
all users' data.
When it starts, the SSH server runs a ``monitor'' process that listens
for connections on port 22. When the monitor receives a connection,
the first thing it does is to spawn a new per-connection worker process
that communicates over the network.
The worker client has no access to the host key or password database---the
worker process can only ask the monitor for help in authenticating.
The monitor-worker API supports a few operations:
\begin{itemize}
\item a \emph{signing} operation, that instructs the monitor to sign a protocol
        transcript, 
      \item a \emph{password-authentication} operation, that instructs
the monitor to check a password (this can be rate limited to prevent
    password guessing), and 
      \item a \emph{start-session} operation, that instructs the monitor
        to create a new process with a shell for the user.
\end{itemize}

\begin{figure}
\begin{verbatim}
[ Host key  ]
[ Passwords ]
[ Port 22   ]
    |
    v
 Monitor --- Spawns per client --> Worker <--- Client1
    |                              Worker <--- Client2
    |                               ...
    --------> Session <----------> Worker <--- ClientN
              process

\end{verbatim}
\caption{The OpenSSH server is split into multiple process to mitigate the compromise of the network-facing code.}
\end{figure}

While this architecture adds quite a bit of complexity to the OpenSSH
server, it has paid off in terms of mitigating the impact of vulnerabilities
in client-facing code.


\section{Example: Web applications}
Companies will often implement Web applications (e.g., a photo-sharing website)
as a number of separate services, running on separate physical machines.
Client connections come into a front-end server that terminates the TLS connection
and proxies client data to a front-end application server.
The front-end server then routes requests to one or more application
services, each of which may have access to different databases.
For example, the login service may have access to the password database,
while the profile service may not.

\begin{figure}
\begin{verbatim}
           [ TLS Key ]
               |                 
               v                  
Client ----> HTTPS  ---> Front-end 
             server       service
                            ^
                            |---> Login   <--> Password DB
                            |     service   
                            |
                            |---> Profile <--> User DB
                            |     service
                            |
                            |---> Photo   <--> Photo DB
                                  service
\end{verbatim}
\caption{Large Web services tend to isolate different application
  features into different services, often on different physical machines.
  The system gates access to these services via minimal front-end client-facing
  servers.}
\end{figure}

\section{Example: Web client}
When you open a PDF attachment in Gmail , you might
worry that the PDF could exploit some bug in your browser that could steal
your sensitive Google data.
To make this kind of attack more difficult, Gmail serves attachments
and other suspect files from a separate domain (``origin')): \texttt{googleusercontent.com}.
Code loaded from \texttt{googleusercontent.com} cannot access cookies
or data for \texttt{google.com}. In this way, even if an attacker can somehow
run JavaScript in your browser, it cannot easily steal your Google cookies.

\section{Example: Web browser}

Web browsers today are extraordinarily complicated pieces of software.
The sensitive data that a browser is trying to protect are things, such as
user cookies, cached data, browser history, and other user data.
The browser may spawn new processes to handle rendering for each site
from each distinct domain/origin.
In this way, if an attacker from one origin can exploit a bug in the
JavaScript engine, the attacker may still not be able to compromise
sensitive user data from other domains/origins.
GPU code, which is extremely complicated and bug-prone, may run in 
yet another process.
Today, compromising a browser entirely often requires finding and exploiting
a collection of bugs in multiple components.
\begin{figure}
\begin{verbatim}
 codec                codec
   ^                    ^
   |                    |
   v                    v
mit.edu             nytimes.com
   |                    |
   |                    | 
   ---- Browser core <------> UI <---> GPU
            |
            v
     [   Cookies   ]
     [ Cached data ]
\end{verbatim}
\caption{Web browsers may isolate the execution of each origin's 
  code in a separate process. They further isolate complicated
  and bug-prone codecs and GPU code in separate processes.}
\end{figure}

\section{Example: Payment Systems}
Processing credit-card transactions in web applications is risky:
if a vendor suffers a compromise, the credit-card network may fine
them or kick them off the network.
To avoid ever having to handle credit-card data, most websites 
use an external payment-processing service that handles credit-card information.
When the user makes a purchase, the vendor redirects the user to the 
payment-processing service, who collects the user's credit-card data.
After payment, the payment-processing service redirects the user
back to the vendor's website.

\begin{figure}
\begin{verbatim}
   --------> Web app -----> Order DB
   |            ^
   |            |
Client    Payment data 
   |            |
   |            |
   --------> Payment -----> Visa/MC
             service   
                | 
                v
        [ Credit card #s ]
\end{verbatim}
\caption{Online vendors often use a separate payment processor
    that handles the user's credit-card data.}
\end{figure}
