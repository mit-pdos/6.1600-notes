\chapter{Authenticated Encryption}

We have just constructed an encryption scheme with
weak (CPA) security: one that provided security
given that the adversary could see
\emph{encryptions} of messages of her choice.
The ``gold standard'' security notion for encryption
schemes allows the attacker to receive both encryptions
of messages of its choice 
\emph{and} decryptions of ciphertexts of its choice.
Our security notion then says that 
even an attacker with this power should not be
able to distinguish which of two chosen plaintext
message a given ciphertexts encrypts.
This new and strong notion of security for encryption
schemes is called 
\emph{security against adaoptive chosen-ciphertext attacks}.

\paragraph{Motivation: Chosen-ciphertext security}
It is not clear why chosen-ciphertext security is the right 
security notion to consider. 
In real-life applications, why would we ever allow an attacker
to obtain \emph{decryptions} of ciphertexts of its choosing?
It turns out that, in many settings, attackers can indeed
trick honest parties into decrypting adversarially ciphertexts
and revealing their contents.

As a simple example, imagine a server that receives encrypted
requests from the network, decrypts them, and either:
\begin{itemize}
  \item returns an error if the decrypted request is malformed, or
  \item silently processes the request otherwise.
\end{itemize}
In this case, an attacker can send ciphertexts to the server and
learn information about their decryptions by noticing whether or not
the server returned an error message.
\marginnote{If the server returns an error when the decrypted message
is illformed, we often call it a \emph{padding oracle}.
Many many real-world protocols using non-chosen-ciphertext-secure encryption
schemes fall victim to this sort of attack.}

CPA-secure encryption schemes provide \emph{no} security guarantees
in this setting. Even if the server leaks a single bit to the attacker 
about the
decrypted value (such as whether the decrypted ciphertext is a
well-formed request or not), the attacker could potentially learn 
the entire secret key!

Chosen-ciphertext security guarantees that such attacks are ineffective.


\section{Defining Authenticated Encryption}

Authenticated-encryption schemes simultaneously provide message 
integrity (as a MAC does) and confidentiality (as CPA-secure encryption does).
Additionally, authenticated-encryption schemes remain secure even when 
an attacker can see encryptions and decrypts of ciphertexts of her choice.
Perhaps unsurprisingly, the standard way to construct an authenticated-encryption
scheme is to combine a CPA-secure encryption scheme with a MAC in a careful way.

We now formally define our strong security notion:
\emph{indistinguishability under chosen ciphertext attacks}, 
also known as IND-CCA2 security or CCA security.

\begin{definition}[CCA security for encryption (strong)]\label{defn:cca}
An encryption scheme is \emph{secure against 
adaptive chosen-ciphertext attacks}
if every efficient adversary wins the following 
game with probability at most $\tfrac{1}{2} + \text{\textquote{negligible}}$:

  \begin{itemize}[noitemsep]
		\item The challenger samples $b \rgets \bin$ and $k \rgets \calK$.
    \item The adversary can make either of the following queries to the challenger
      repeatedly:
      \begin{itemize}
    \item \emph{Chosen-plaintext queries}
          \begin{itemize}
            \item The adversary sends the challenger a message $m_i \in \calM$
						\item The challenger replies with $c_{m_i} \gets \Enc(k, m_i)$.
          \end{itemize}
    \item \emph{Chosen-ciphertext queries}
          \begin{itemize}
						\item The adversary sends the challenger a \emph{ciphertext} $c_j \notin \set{c_{m_0}, \ldots, c_{m_i}}$
						\item The challenger replies with $m_{c_j} \gets \Dec(k, c_j)$.
          \end{itemize}
      \end{itemize}
    \item The adversary then sends two messages $(m^*_0, m^*_1) \in \calM^2$ to the challenger,
          where $\abs{m^*_0} = \abs{m^*_1}$.
    \item The challenger replies with $c^* \gets \Enc(k, m^*_b)$.
    \item The adversary can make more chosen-plaintext queries
      and more chosen-ciphertext queries. (The adversary may not
      make a chosen-ciphertext query on the challenge ciphertext $c^*$.)
		\item The adversary outputs a value $b' \in \bin$.
    \item The adversary wins if $b=b'$. 
	\end{itemize}
\end{definition}


\subsection{Encrypt then MAC}
We typically achieve CCA security using the ``encrypt-then-MAC'' construction:
\begin{itemize}
  \item First, encrypt the message using a CPA-secure encryption scheme on key $k_\mathsf{Enc}$.
  \item Next, MAC the \emph{ciphertext} using a secure MAC scheme and an independent key 
    $k_\mathsf{MAC}$.\marginnote{As we discuss below,
    it is possible to derive both keys $k_\mathsf{Enc}$
    and $k_\mathsf{MAC}$ from a single key $k$ using a pseudorandom function.}
  \item Output the ciphertext and the MAC tag.
\end{itemize}
The decryption routine first checks the MAC tag, then decrypts the ciphertext.

Using independent keys $(k_\mathsf{Enc}, k_\mathsf{MAC})$ is important 
in encrypt-then-MAC, as in many other cryptographic constructions.
For example, a CPA-secure encryption scheme using $n$-bit keys can reveal
the low order $n/2$ bits of its secret key in the ciphertext.
And a secure MAC scheme using $n$-bit keys can reveal the high-order
$n/2$ bits of its secret key in each MAC tag.
Used independently, the encryption scheme and the MAC scheme are both
secure.
Used together with the same key $k$, the attacker learns all $n$ bits
of the key $n$ and can break both primitives!

So, in general, you should always use independent keys for different primitives.
To reduce the amount of keying material parties need to store, it is actually 
sufficient to store a single secret key $k$ for a pseudorandom function $F$
and derive all subsequent keys from the pseudorandom strings $F(k, 0), F(k, 1), \dots$.

\begin{theorem}[Informal]
The Encrypt-then-MAC construction yields a CCA-secure encryption scheme,
provided that: the underlying encryption scheme is CPA-secure and the
underlying MAC scheme is secure (existentially unforgeable against adaptive chosen message attacks).
\end{theorem}

\paragraph{Warning! \textbf{Only} use encrypt-then-MAC}\marginnote{Reminder:
You should never need to implement authenticated-encryption schemes yourself.
Instead use an off-the-shelf implementation that does the hard work for you.
AES-GCM is one popular and widely implemented authenticated encryption scheme
}
There are a number of bad ways to combine encryption and MACs to attempt to
build authenticated-encryption schemes.
MAC-then-encrypt is one way. Encrypt-and-MAC (i.e., MAC the message
instead of the ciphertext) is another.
Neither of these constructions is necessarily CCA-secure when used with a
CPA-secure encryption scheme and a secure MAC scheme.
So the \emph{\textbf{only}} flavor of authenticated encryption you should use
is encrypt-then-MAC.

\section{AES-GCM (Galois Counter Mode)}\label{sec:enc:gcm}
One of the most widely used authenticated-encryption constructions is AES-GCM.
It follows the encrypt-then-MAC paradigm.
It uses AES as a pseudorandom function for counter-mode encryption (\cref{sec:enc:ctr}).
It uses a Carter-Wegman-style MAC (\cref{sec:mac:cw}) as the MAC scheme.

There are a few optimizations that AES-GCM uses beyond what we have described:
\begin{itemize}
  \item AES-GCM derives both the encryption and MAC keys from a single short key
        using a pseudorandom function.
  \item AES-GCM implements a fast form of the Carter-Wegman MAC that does not 
        need arithmetic modulo a big prime $p$, as the scheme described in \cref{sec:mac:cw} does.
        Instead, of defining the MAC using $\Z_p$ (integers mod a 128-bit prime $p$),
        GCM works with 128-bit strings.
        The GCM mode of operation replaces addition modulo $p$ with XOR of 128-bit strings
        and it replaces multiplication modulo $p$ with a somewhat complicated operation
        on 128-bit strings.\marginnote{If you are interested, to implement the
        multiplication operation: think of both 128-bit strings as polynomials with 128
        coefficients in $\Z_2 = \{0,1\}$. Multiply the polynomials, reducing the 
        coefficients modulo $2$. Then reduce the resulting polynomial modulo some 
        fixed polynomial of degree-$128$. Then interpret the result as a $128$-bit string.} 
        (Formally, the scheme works over the field $\F_{2^{128}}$ of order $2^{128}$.)
        This gives a big performance boost with no loss in security.
\end{itemize}


% TODO: finish
% TODO: diagram is pretty important here

% test edit
