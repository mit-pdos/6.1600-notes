\chapter{Privacy and zero-knowledge proofs}

So far, we have discussed several cryptography primitives, from hash
functions to encryption schemes, and explored many applications of those
primitives to systems security.  These primitives have provided security,
but in a very all-or-nothing sense: in order to provide broadly applicable
security, our definitions required that a certain party (with the key)
could either completely decrypt the message, learning the message
contents, or cannot do anything with the message at all.

\section{Zero-knowledge proofs of knowledge}

Zero-knowledge proofs allow one party (called a prover) to convince
another party (called a verifier) that the prover ``knows'' some secret,
without revealing anything else about the secret.
More formally, say that both parties hold a function $f$, and a value $y = f(x)$.
The prover may want to convince the verifier that it ``knows'' an $x$ such that
$y=f(x)$.
We call such a prover-verifier interaction a \emph{proof system}.
If the prover reveals no information about $x$, apart from the fact that $y=f(x)$,
to the verifier, we say that the proof system is a \emph{zero-knowledge proof of knowledge}.

The proof takes the form of an interaction
between the prover and the verifier, in which
the two parties exchange a sequence of messages.
At the end of the interaction, the verifier either
accepts the proof or rejects it.
This is an \emph{interactive proof}: the prover and
verifier may exchange many messages.
In contrast, in a classical (non-interactive) proof,\marginnote{Probably
all of the proofs you have seen in your life up until this point,
including the ones in your math textbooks, are classical proofs.}
the prover writes down a proof and sends it to the verifier in a single message.

In our example, both parties hold a function $f$ and a value $y$.
The prover wants to convince the verifier that it knows a value $x$ such that $y=f(x)$.\marginnote{The
literature usually refers to the secret value $x$ as the \emph{witness}.}
To be a zero-knowledge proof of knowledge,
the proof system must have three properties.
We state this informally:
\label{sec:zkdefn}
\begin{itemize}
\item \emph{Completeness.} If the prover ``knows'' $x$,
      it will always convince an honest verifier to accept.

\item \emph{Soundness.} If the verifier accepts the
      proof, then prover must really ``know'' $x$.
      The soundness property is about protecting
      an honest verifier from a cheating prover:
      if the prover does not know $x$, then no matter
      how it cheats it will not be able to convince
      the verifier that it does.

\item \emph{Zero-knowledge.} The verifier 
      ``learns nothing'' about $x$, apart from the fact that $y=f(x)$,
      as a result of interacting with the prover.
      The zero-knowledge property is about protecting
      the prover from a malicious verifier: if the prover
      is honest, then no matter how the verifier misbehaves,
      it should learn nothing about the prover's secret witness $x$.
\end{itemize}
\hcg{Insert formal definitions here?}

One important question here is: How do we define the notion of knowledge?  
That is, what does it mean for the prover to really ``know'' $x$?
A na\"ive definition might say that $x$
is stored somewhere in the prover's memory, but that might be too strong:
a prover might ``know'' $x$ without storing it literally in its memory,
perhaps storing it in base64-encoded form.
A more meaningful definition of knowledge, which is the one we will use,
thinks of knowledge in terms of extractability.
That is, we say that a prover knows $x$ if there is an efficient 
algorithm that extracts $x$ by interacting with the prover and 
observing the prover's internal state.

We also must pin down what it means for the verifier to
learn nothing through an interaction with the prover.
A good way to define this, it turns out, is to think of
``learning nothing'' in terms of the verifier being able to simulate its
interaction with the prover without actually talking to the real prover.
Specifically, if the verifier can produce a transcript of its hypothetical
interaction with the prover---that is cryptographically indistinguishable 
from a real transcript---without actually communicating with the
prover, we say the verifier has learned nothing about $x$.
\marginnote{
  There is a rich theory of zero-knowledge proof systems that we
  will not be able to cover here.
  One surprising fact is that there exist zero-knowledge proofs of knowledge
  for \emph{any} choice of the public function $f$, provided that it is
  an efficiently computable function.
}

One immediate application of zero-knowledge proofs of knowledge is to authentication.
The prover's witness $x$ becomes the user's secret key,
the verifier's $y$ is the user's public key, and the
function $f$ is a public parameter of the authentication scheme.
The authentication protocols we've seen so far, such as challenge-response protocols using MACs or signatures,
do not quite satisfy the notions of soundness and zero-knowledge we use here.
For example, in a MAC-based challenge-response protocol, 
the server learns the MAC of a
server-chosen challenge value under the client's (prover's) secret key.
Provided that the MAC is secure, the verifier could not 
produce a simulated transcript that is identical to a
real one without actually interacting with the real prover.

\section{Discrete-log problem and Schnorr signatures}

As an example of a zero-knowledge proof of knowledge, we will show
how a prover can convince a verifier that it knows a discrete
log, without revealing it. 

\section{Reminder: The discrete-log problem}
We recall the discrete-log problem.
The problem is parameterized by 
a group $\G$ of prime order $q$,
generated by a value $g \in \G$.\marginnote{In cryptographic settings, $q$ is a big prime.}
Then, given a group element $y = g^x \in \G$, for $x \getsr \Z_q$,
the discrete-log problem is to find the value of $x\in \Zq$.
We have already seen the discrete-log problem in
the context of Diffie-Hellman key exchange and elliptic-curve signatures.


\section{Schnorr's protocol for proving knowledge of discrete logs}

In Schnorr's protocol, the prover and verifier both hold the
generator $g \in \G$ and the group order $q$.
In addition the prover holds a secret value $x \in \Z_q$, the verifier
holds a public value $y \in \G$.
The prover must convince the verifier that it knows the discrete log of $y$
base $g$: that is, that it knows $x$ such that $y=g^x \in \G$.
The prover wants to prove this to the verifier this without revealing
anything about else about its secret value $x$.

The protocol goes as follows:
\begin{itemize}

\item The prover picks a random $r\in \Z_q$, and sends $R=g^r\in \G$ to the verifier.

\item The verifier picks a challenge bit $c$ at random, either 0 or 1,
      and sends the challenge to the prover.

\item The prover sends back a response $z \in \Z_q$, chosen as follows: \marginnote{In
  the true Schnorr protocol, the verifier samples $c \getsr \Z_q$, as we
    discuss in \cref{sec:hvzk}}
  \begin{itemize}
    \item If $c=0$, the prover sets $z \gets r \in \Z_q$.
    \item If $c=1$, the prover sets $z \gets r+x\in \Z_q$.
  \end{itemize}
  Mathematically, $z=r+cx\in \Z_q$.

\item The verifier checks that $g^z=R y^{c}\in \G$.  If equal, the verifier accepts,
  otherwise the verifier rejects.

\end{itemize}

Why is interaction necessary here?
It turns out that this protocol critically depends on the prover 
not knowing the challenge bit $c$
in advance.  If the prover knew $c$ in advance, the prover could pick
$z \getsr \Z_q$ at random, and just compute $R$ as $(g^z)(y^c)^{-1}\in \G$.

\paragraph{Reducing soundness error.}
In any given interaction of the protocol, the prover could falsely
convince the verifier that it knows $x$, using the attack we just
described. The prover only will succeed in this attack with probability $\frac 12$,
which is exactly the probability that the prover guesses the verifier's challenge
$c$ in advance.
By repeating this protocol $\lambda$ times and accepting only if all $\lambda$
iterations accept, the verifier
can reduce the probability of accepting a cheating prover (called the ``soundness
error'') to $2^{-\lambda}$.

A more efficient way to reduce the soundness error is to have 
the verifier sample the challenge value $c \getsr \Z_q$ as a
random $\Z_q$ element instead of a random bit.
With this transformation, the soundness error becomes roughly $1/q$,
which is $\approx 2^{-256}$ for the cryptographic groups we use in practice.
An important caveat is that if we increase the challenge space in this way,
we cannot prove that the resulting protocol is zero knowledge.
This modified protocol only
satisfies a weaker flavor of zero knowledge called ``honest-verifier zero knowledge.''
We won't discuss those details here.



\section{Analysis of Schnorr's protocol}

We will now argue that Schnorr's protocol is really
a zero-knowledge proof of knowledge of discrete log.
To do this, we must show that the protocol satisfies
completeness, soundness, and zero knowledge, as
we defined them in \cref{sec:zkdefn}.

The completeness follows by construction.
The more challenging steps are proving soundness and zero knowledge.

\subsection{Soundness using an extractor}

We first need to convince ourselves that the protocol is \emph{sound}: that is,
if the verifier accepts, the prover really knows $x$.  We can prove
soundness by showing that, if there is a (possibly adversarial) prover
$P^*$ that can convince our honest verifier to accept, there
exists an efficient extractor that can use the cheating prover $P^*$
to extract $x$ with some high probability (say, $\gg 1/2$).

For simplicity, assume here that the prover $P^*$ manages to convince
the verifier to accept with probability $1$---i.e., for all choices of
the verifier's challenge bit $c$ and all choices of the prover's randomness.
Then, given a prover algorithm $P^*$,
the extractor for Schnorr's protocol works as follows:

\begin{itemize}

\item The extractor runs the prover $P^*$ to get some initial protocol transcript with challenge $c=0$.
The transcript consists of $(R, c=0, z)$.

\item The extractor rewinds the state of the prover $P^*$ to just after the point when it
      sent the first protocol message $R$ to the verifier.

\item The extractor runs $P^*$ again, but this time feeds its the challenge $c=1$.
This produces a second transcript $(R, c=1, z')$.

\item The extractor outputs $z'-z$ as the discrete log $x$.

\end{itemize}

By our assumption---namely, that $P^*$ can convince our honest verifier
to accept with probability one---both
runs of the prover $P^*$ by the extractor must convince the verifier to accept. (Otherwise,
the prover $P^*$ would not convince the verifier in at least one of the two runs of the protocol.)
That means that both $g^z=R \in \G$ and that $g^{z'}=RX \in \G$.  Thus, $g^{z'-z}=y \in \G$,
which is exactly the definition of what it means for something ($z'-z$
in particular) to be the discrete log of $y \in \G$.

Here we have glossed over the details of what happens if the prover
convinces the verifier with some non-negligible probability that is
nonetheless less than 1.
A similar but more involved argument handles that case.

\subsection{Zero-knowledge using a simulator}

Our second task is to show that Schnorr's protocol satisfies zero knowledge.
That is, that even a malicious verifier $V^*$ that deviates from
the protocol ``learns nothing'' about the prover's witness
$x$ as a result of interacting with the prover in the Schnorr protocol.
To do this, given a malicious verifier algorithm $V^*$, we construct a \emph{simulator} algorithm.
The simulator's job
is to produce a transcript of the prover-verifier interaction that is
indistinguishable from a true prover-verifier interaction.
Crucially, in the simulated interaction, the simulator \emph{does not know} the witness $x$,
while in the real prover-verifier interaction, the prover \emph{does know} the witness $x$.

Given the verifier $V^*$, 
we construct this simulator as follows:

\begin{itemize}

  \item Make a guess $c' \getsr \{0,1\}$ of the verifier's challenge bit.

  \item Choose the third protocol message $z \getsr \Z_q$ at random.

  \item Set the first message to $R \gets g^z y^{-c'} \in \G$.

  \item Run the verifier $V^*$ on first message $R$.
        The verifier outputs a challenge $c$.
  \item If $c=c'$, output $(R, c, z)$ as the simulated transcript.
        This happens with probability $1/2$.

\item Otherwise, retry.  This happens with probability $1/2$ as well.

\end{itemize}

Both the extractor and simulator constructions rely crucially on being
able to rewind the prover or verifier and try again. 
In a real prover-verifier interaction, the verifier cannot rewind the prover
and the prover cannot rewind the verifier.
This is why the existence of an extractor does \emph{not} mean that
in a real protocol interaction the verifier can extract the witness
$x$ from the prover.


\section{Fiat-Shamir heuristic and Schnorr signatures}
\label{sec:hvzk}

One challenge with Schnorr's protocol as we have presented it,
and zero-knowledge proof systems more generally,
is that they are interactive.
They require the prover and verifier to send messages back and forth.
A clever trick, called the Fiat-Shamir heuristic, allows us to turn interactive
zero-knowledge proofs into non-interactive proofs, provided that
\begin{itemize}
  \item we use a cryptographic hash function that we model as a random oracle, and
  \item the verifier only sends the prover independent random values.\marginnote{In Schnorr's protocol,
    the verifier's challenge is just a random bit, so it satisfies this second property.}
\end{itemize}
The idea is that, whenever the protocol expects the verifier to send a challenge to the
prover, we can instead derive the challenge just as a hash of the protocol transcript so far.
The prover can then execute the zero-knowledge
protocol against a synthetic verifier (the hash function), and produce
a transcript.  The prover can now send this transcript to the verifier,
and if the verifier can confirm that indeed all of the challenges were
correctly computed using a cryptographic hash function, such as 
SHA-256, it's sound to accept this transcript as a proof.

We can use this Fiat-Shamir heuristic to construct a (regular,
non-interactive) signature scheme from an (interactive) zero-knowledge
proof of knowledge for the discrete-log problem.
The elliptic-curve signatures used in practice today
effectively take the Schnorr protocol for proof of knowledge of discrete log
using challenge space $\Z_q$, and apply the Fiat-Shamir heuristic to it.
The resulting signature scheme is called the \emph{Schnorr signature scheme}.
