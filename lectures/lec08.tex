\chapter{Introduction to Encryption}

So far, we have explored methods for authenticating data---verifying that it has not been modified in transit.
However, the integrity-protection mechanisms we
have discussed provide no \emph{confidentiality}:
a network eavesdropper can still view everything that 
we send over the network.

In this chapter, we will discuss \emph{encryption},
which allow two parties to exchange messages over an insecure
network while hiding the contents of their communications.

% TODO: demo of tcpdump output would be cool

We will cover encryption in a sequence of steps:
\begin{itemize}
  \item First, we will construct an encryption scheme with a \emph{weak form of security} for \emph{fixed-length messages} for settings in which the sender and recipient have a \emph{shared secret key}.
  \item Next, we will show how to extend this scheme to support \emph{variable-length messages}.
  \item Then, we will show how to improve the scheme to have a \emph{strong form of security}.
  \item Finally, we will show how to implement encryption in settings in which the sender and recipient \emph{have no shared secrets}.
\end{itemize}
At the conclusion of this part, we will discuss how deployed systems
use encryption, and 
we will think about some problems that encryption does not solve.

\section{Background}

\paragraph{The need for encryption}
The internet is a massive network of wifi access
points, routers, switches, undersea cables, DNS
servers, and much more.
There are many, many
devices for a potential adversary to compromise
and many vantage points from which an attacker can 
observe network traffic. Every single hop your
packets take is a potential point of compromise.

To make matters worse, most standard network protocols provide \emph{no authentication or encryption}: in Ethernet, IP, DNS, 
email, HTTP, and others, an adversary is free to modify and read the traffic we send and receive. 
Protecting confidentiality typically requires augmenting these standard network
protocols with some form of authentication and encryption.

\paragraph{Systems using encryption}
Encryption shows up in a large number of deployed systems. A few examples are:
\begin{itemize}
  \item \textbf{Messaging apps}, such as WhatsApp, Signal, and iMessage, encrypt
        traffic between app users so that the server cannot easily read it.
  \item \textbf{Network protocols}, such as SSH and HTTPS use encryption to 
        protect traffic between a service's clients and servers.
  \item \textbf{File-storage systems} use encryption to protect data at rest.
        So if a thief steals your laptop, they will not easily be able to read
        the encrypted files on your hard disk.
\end{itemize}

\section{Encryption Scheme Syntax}
Encryption schemes are defined with respect to a key space $\calK$, a message space $\calM$, and a ciphertext space $\calC$.
For now, think of $\calK = \calM = \zo^n$ and $\calC = \zo^{2n}$,
for a security parameter $n$.\marginnote{Remember that in practice,
we will often take the size of the keyspace $\abs{\calK}$ to be at least
$2^{128}$ to prevent brute-force key-guessing attacks.} 
An encryption scheme then consists of two algorithms:

\begin{itemize}
	\item $\Enc: \calK \times \calM \rightarrow \calC$
	\item $\Dec: \calK \times \calC \rightarrow \calM$
    \marginnote{As we will see later on, the decryption algorithm 
    $\Dec$ can sometimes output \textquote{FAIL} or $\bot$.}
\end{itemize}

The definition of correctness for an encryption scheme just states
that if you encrypt a message $m$ with a key $k$, and then you decrypt
the resulting ciphertext with the same key $k$, you end up with the 
same message $m$ that you started with:
\begin{definition}[Encryption Scheme, Correctness]
An encryption scheme is correct if, for all keys $k \in \calK$ and all messages $m \in \calM$, $\Dec(k, \Enc(k, m)) = m$.
\end{definition}

Defining security for encryption scheme is tricky business.
Many of the most obvious security definitions are insufficient:
\begin{itemize}
  \item \textbf{Bad definition:} \emph{``An encryption scheme is secure 
        if it is infeasible for an attacker to recover the plaintext
        message given only a ciphertext.}
        This definition admits encryption schemes in which the ciphertext
        leaks half of the plaintext bits.
  \item \textbf{Bad definition:} \emph{``An encryption scheme is secure 
        if it is infeasible for an attacker to recover any bit of
        the plaintext message given only a ciphertext.}
        This definition admits encryption schemes in which the ciphertext
        leaks the parity of the plaintext bits.
\end{itemize}

The starting-point (weak) security definition we will use
is called \emph{indistinguishability under adaptive chosen plaintext attack}
($\INDCPA$).
Intuitively, a scheme is CPA-secure if an attacker cannot tell which of two chosen messages are encrypted, even after seeing encryptions of many attacker-chosen messages.

\begin{definition}[Encryption Scheme, CPA Security (weak)]\label{defn:cpa}
Formally, an encryption scheme $(\Enc, \Dec)$ over message space $\calM$
and key space $\calK$ is CPA-secure if 
all efficient adversaries win
the following game with probability at most $\tfrac{1}{2}$ + ``negligible:''
  \begin{itemize}[noitemsep]
		\item The challenger samples $b \rgets \bin$ and $k \rgets \calK$.
    \item Polynomially many times: \quad \emph{// Chosen-plaintext queries}
          \begin{itemize}
            \item The adversary sends the challenger a message $m_i \in \calM$
            \item The challenger replies with $c_i \gets \Enc(k, m_i)$.
          \end{itemize}
    \item The adversary then sends two messages $m^*_0, m^*_1 \in \calM$ to the challenger.
          (We require $\abs{m^*_0} = \abs{m^*_1}$.)
          \marginnote{Standard encryption systems do not hide the
          length of the message being encrypted.
          So, if the message space $\calM$ contains messages of different
          lengths, our security definition requires the adversary to distinguish
          the encryption of two messages of the \emph{same} length.}

    \item The challenger replies with $c^* \gets \Enc(k, m^*_b)$.
		\item The adversary outputs a value $b' \in \bin$. The adversary wins if $b=b'$. 
	\end{itemize}
\end{definition}

One potentially surprising consequence of the CPA-security definition 
for encryption schemes is:
\begin{center}
\emph{For an encryption scheme to be secure in any meaningful sense,\\
the encryption algorithm \textbf{must} be randomized}.
\end{center}\marginnote{In contrast, secure MACs can be---and typically are---deterministic!}

If the encryption algorithm is deterministic (i.e., not randomized),
an attacker can win the CPA security game in \cref{defn:cpa}
with probability 1.
To do so:
\begin{itemize}[noitemsep]
  \item The attacker first asks for encryption of a message $m_0$
and receives a ciphertext $c_0$.
  \item Then, the attacker attacker choose a message $m_1 \neq m_0$ and
sends $(m^*_0, m^*_1) = (m_0, m_1)$ to the challenger
and receives the challenge ciphertext $c^*$.
  \item If $c^* = c_0$, the attacker outputs 0.
Otherwise, the attacker outputs 1.
\end{itemize}

Deterministic encryption schemes are not only broken in theory,
they are also broken in practice.
For example, if you encrypt the pixels of an image using a deterministic
encryption scheme, the encrypted image essentially reveals the plaintext image.

\paragraph{Why CPA security is a ``weak'' form of security}
There are two reasons why CPA security is a weak or insufficient 
definition of security for an encryption scheme:
\begin{itemize}
  \item First, the CPA security definition
    guarantees nothing about message
    \emph{integrity}: an attacker can modify the
    ciphertext and potentially change the meaning
    of the encrypted message in a meaningful way
    (even if the attacker does not know the encrypted message!).
  \item Second, the CPA security definition guarantees
    nothing in the event that the adversary can obtain
    \emph{decryptions} of ciphertexts of its choosing.
    In practice, attackers can often obtain decryptions
    of chosen ciphertexts.
    for example, in a system where a client sends encrypted messages to a server and the server does something in response, an attacker can send encrypted queries to the server and observe its behavior to learn some function of the decrypted contents of the message. Later on, we will expand our definition to include these \textit{chosen ciphertext attacks}.
\end{itemize}

\section{One-time Pad}
The one-time pad is perhaps the simplest encryption scheme.
Its keyspace, message space, and ciphertext space are all the set of $n$-bit strings.
The algorithms are then:
\begin{itemize}
  \item $\Enc(k, m) \to c$. Compute $c \gets (k \oplus m)$.
  \item $\Dec(k, c) \to m$. Compute $m' \gets (k \oplus c)$.
\end{itemize}

The encryption scheme is correct since $\Dec(k, \Enc(k, m)) = k \oplus (k \oplus m) = m$.
A less obvious fact is that it is also \emph{one-time secure}:
if an attacker sees only one message with encrypted a one-time-pad key $k$,
it learns nothing about the underlying plaintext.
\marginnote{The one-time pad is actually one-time secure in a very strong sense:
it protects confidentiality even against a computationally unbounded attacker---one
that can perform arbitrary amounts of computation.}

\paragraph{The ``two-time pad'' attack}
The one-time pad is insecure if the same key $k$ is every used to 
encrypt two messages.
In particular, if two ciphertexts are ever computed using the same key, we
have:
\begin{align*}
  c_1 &= k \xor m_1 \\
  c_2 &= k \xor m_2 \\ 
  c_1 \oplus c_2 &= (k \xor m_1) \xor (k \xor m_2) = m_1 \xor m_2.
\end{align*}
The attacker then learns the XOR of the two encrypted messages,
which is often enough to leak all sorts of sensitive information
about the plaintext.
For example, if the attacker knows some bits of $m_1$, it can learn
some bits of $m_2$.

\paragraph{Why the one-time pad is useful}
The one-time pad encryption scheme feels in some sense useless: if a secure channel exists through which Alice and Bob can exchange an $n$-bit secret key, they may as well use that channel to exchange the message itself!
There is some merit still in the one-time pad: it is possible
to exchange the key ahead of time, and then send encrypted messages
later on.
Diplomats indeed used the one-time pad in this way throughout the 20th century.
They would exchange huge books of keying material (random strings) 
and then use these keys to communicate securely over long distances.

In practice, it is much more convenient to be able to exchange a \emph{short}
key and then use it to encrypt many \emph{long} messages.

\section{A Weak Encryption Scheme}\label{sec:lec8:weakenc}
What we effectively need is a way to generate many bits of random-looking keys 
(i.e., keys for the one-time pad encryption scheme)
from a single short random string.

To generate this, we can use a pseudorandom function! 
\marginnote{If you forget what a pseudorandom function is, refer back to \cref{defn:prf}.}
In particular, we would like a pseudorandom 
function of the form $F \colon \bin^n \times \bin^n \rightarrow \bin^n$. 
\marginnote{In practice, we will use AES or another block cipher
as a pseudorandom function, so we will have $n = 128$ or so.}

Using such a pseudorandom function, we can construct a new 
encryption scheme that replaces the truly random keys 
with pseudorandom keys generated from the pseudorandom function.
The keyspace and message space for this encryption space are $\zo^n$
and the ciphertext space is $\zo^{2n}$.
The algorithms are:
\begin{itemize}
	\item $\Enc(k, m) \to c$:
      \begin{itemize}
        \item Sample a random value $r \rgets \bin^n$.
              We call this the ``nonce.''
        \item Compute a one-time key $k \gets F(k, r)$ using the pseudorandom function.
        \item Use this key to compute the ciphertext using the one-time pad: 
          output $c \gets (r, k \xor m) \in \zo^{2n}$.
      \end{itemize}
	\item $\Dec(k, c)$:
        \begin{itemize}
          \item Parse the ciphertext $c$ into $(r, c')$.
          \item Compute $m \gets c' \xor F(k, r)$.
        \end{itemize}
\end{itemize}
\marginnote{While this encryption scheme is CPA-secure,
it provides no message integrity.
For any string $\Delta \in \zo^n$,
an attacker can modify a ciphetext $(r, c')$ to $(r, c' \oplus \Delta)$.
This ciphertexts now decrypts to the original message XORd
with the attacker-chosen string $\Delta$.
}

Correctness holds by construction.
The security argument goes in three steps:
\begin{itemize}
  \item First, we argue that if $n$ is large enough, the probability 
        that the encryption algorithm ever chooses the same $r$ value twice
        is negligible.
        \marginnote{By the Birthday Paradox, after encrypting $T$ messages,
        the probability of a repeated $r$ value is $O(T^2/2^n)$,
        which is negligible in the key length~$n$.}
  \item Second, we argue that as long as the encryption algorithm
        never chooses the same $r$ value twice, 
        we can replace the values $F(k, r)$ with truly random strings.
        By the security of the pseudorandom function, the adversary will
        not notice this change.
  \item At this point, we can appeal to the security of the one-time pad scheme
        to argue that the adversary has no chance of winning the security 
        game (\cref{defn:cpa}).
\end{itemize}

For security to hold, it is crucial that the probability 
that the encryption algorithm uses the same encryption nonce $r$ twice
be negligibly small.
If the encryption algorithm ever selects the
same random nonce $r$ twice, the pad $F(k, r)$ will be identical in 
two ciphertexts.
An attacker can then apply the two-time-pad attack to recover the XOR of the two messages.

By the Birthday Paradox, if we sample the nonce from a space of size $2^{128}$, we can expect a collision in 128-bit random values once around $2^{64}$ have been generated. Therefore, any single encryption-key must be used for $\ll 2^{64}$ messages.
Cryptographic standards typically limit the number of bytes that users can
encrypt with the same key to prevent these sorts of problems.


\section{Encrypting longer messages: Counter mode}
\label{sec:enc:ctr}

The CPA-secure encryption scheme of \cref{sec:lec8:weakenc} only
allowed encrypting messages of a fixed length.
We now show how to use \emph{counter-mode encryption} to extend
this scheme to support messages of arbitrary length.
That is, we will construct an encryption scheme 
$\Enc \colon \calK \times \zo^* \to \zo^*$
for messages of any length.
\marginnote{In practice, encryption schemes place some
(large) bound on the length of encrypted messages.
For example, the AES-GCM cipher has a maximum message
limit of just under 64 GiB.
}

Counter-mode encryption works much as the CPA-secure encryption
scheme we have already seen in \cref{sec:lec8:weakenc}, except 
that we split the message into blocks and encrypt each block separately.

\newcommand{\ToString}{\mathsf{ToString}}
We will use the function $\ToString \colon \{0, \dots, 2^n-1\} \to \zon$,
which converts an integer in $\{0, \dots, 2^n -1\}$ to an $n$-bit string in the natural way.

The encryption scheme uses a pseudorandom function $F \colon \calK \times \zon \to \zon$.
The secret encryption key is a key $k \in \calK$ for the pseudorandom function.
To encrypt a message, we choose a fresh random value $r \getsr \{0, \dots, 2^n-1\}$ and 
XOR the $i$th block of the message with the value $F(k, \ToString(r+i \bmod 2^n))$.

\begin{itemize}
  \item $\Enc(k, m):$
    \begin{itemize}
      \item Split the message $m$ into blocks of $n$ bits: $(m_1, m_2, m_3, \dots, m_\ell)$.
            The last message block $m_\ell$ may be shorter than $n$ bits.
      \item Sample a random nonce 
            $r \getsr \{0, \dots, 2^n-1\}$.
        \marginnote{The nonce is sometimes called an ``initialization vector'' or ``IV.''} 
      \item For $i = 1, \dots, \ell$: Compute $c_i \gets m_i \oplus F(k, \mathsf{ToString}(r+i \bmod 2^n))$.\\
            (If the last message block $m_\ell$ is less than $n$ bits long, 
            truncate the last ciphertext block $c_\ell$ to the length of $m_\ell$.)
      \item Output the ciphertext $c = \big(r, c_1, \dots, c_\ell)$.
    \end{itemize}

  \item $\Dec(k, c):$
    \begin{itemize}
      \item Parse the ciphertext $c$ as $(r, c_1, \dots, c_\ell)$,
            where all values but the last are exactly $n$ bits long.
          \item For $i = 1, \dots, \ell$: Compute $m_i \gets c_i \oplus F(k, \ToString(r+i \bmod 2^n))$.
            (Truncate $m_\ell$ to the length of $c_\ell$.)
      \item Output the message $m = (m_1 \| \dots \| m_\ell)$.
    \end{itemize}

\end{itemize}

As long as the space of random values $r$ is large enough to ensure that
the encryption routine never evaluates the pseudorandom function $F(k, \cdot)$
on the same input twice, this scheme will be CPA secure.

You may notice that this encryption scheme reveals
the length of the encrypted message to the attacker! This
indeed is a potential risk, but in some sense it
is required: if we were to hide the length of the
message, we would need to set some maximum message
length and pad it up to this length. If we did
this, encrypting a single word would necessarily
result in a ciphertext equal in length to the
ciphertext of encrypting a movie! This would
greatly decrease the practicality of our
encryption scheme. For applications where hiding
the length is especially important, the messages
can be padded to ensure they are all the same
length before they are passed to the encryption
scheme.

