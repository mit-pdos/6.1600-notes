\chapter{Software Trust}

The central question of this chapter is: 
\begin{quote}
\emph{How do we know whether a system is running the 
  software that we expect it to be running?}
\end{quote}
This question comes up both when we are interacting with a machine 
in person (e.g., typing a passcode into our phone) or across a network
(e.g., when sending sensitive data to a far-away server).
\marginnote{A separate question, which we will discuss in future chapters
is: How do we ensure that the software itself is ``good'' or ``bug-free?''}

%When we interact with computers, we need some way of knowing that they are running real software. We would like to know this by minimizing our reliance on \emph{trust}.

\paragraph{Threats to software integrity.}
There are a number of threats that might cause a machine to
be running unexpected software:
\begin{compactitem}
\item \textbf{Malware.} 
      An adversary may install bad software onto the laptop, such as a keylogger.
\item \textbf{User error.} A user may inadvertently install malware onto a
machine. \marginnote{A classic way to trick users into installing malware is
  to show them a warning (e.g., on a webpage) that says ``Your computer is infected. Please download
  and install this anti-virus software.''}
\item \textbf{Software supply-chain problems.} 
      An adversary may inject malware into a real app's libraries, by tricking or coercing a developer.
\item \textbf{Malicious updates.} An adversary may trick the software update process, converting a real piece of software into a malicious one.
\end{compactitem}


\paragraph{The software supply chain.}
There are many steps that take place between the
development of a piece of software and its use:
\begin{compactenum}
	\item Developers write code.
        Their code may use many third-party libraries.
	\item Compilers build and package the application. 
	\item The software vendor distributes binaries over the network.
	\item Users download new software.
	\item Users download software updates.
	\item Users launch applications on their devices.
	\item Running applications interact with remote servers running code.
\end{compactenum}

\section{Library Imports}
% TODO: summary of dependency vulnerabilities, etc

\subsection{Example: Python Imports}

In Python, importing a library requires downloading a library
using the \texttt{pip} package manager:
\begin{lstlisting}	
	pip install requests
\end{lstlisting}
After that a user can import the package into their code like this:
\begin{lstlisting}	
	import requests
\end{lstlisting}

Behind the scenes, the PyPi service maintains a database
that maps package names (e.g., \texttt{requests}) to 
a piece of code.
When you type \texttt{pip install requests}, the \texttt{pip}
program fetches the code from the PyPi repository and
installs it on your machine.

\paragraph{Benefits} Benefits of this approach are:
\begin{itemize}
  \item The centralized service makes it easy for users 
        to identify packages.
  \item It is relatively easily for users to discover
        and download software updates.
  \item Developers do not need to run their own 
        code-distribution service.
\end{itemize}

\paragraph{Downsides} Downsides of this approach are:
\begin{itemize}
  \item The centralized update service is a single point of 
        failure: if an attacker is able to change the code in
        PyPi, it can infect a large number of machines at once.
  \item The end user has no idea who actually produced the library
        code. The user only is able to specify the package name.
  \item In Python, the naming scheme is ambiguous: if there is
        a public package and a private package both with the name
        \texttt{requests}, it's not clear when a user imports
        \texttt{requests}, which one the user wants to import.
\end{itemize}


\subsection{Example: Go Library Imports}
The Go programming language takes a slightly different
approach to package management.
In the Go programming language, a user
imports a library/package by 
specifying the URL of the package's 
Git repository.
For example, an import might look like this:
\begin{lstlisting}	
	import "github.com/grpc/grpc-go"
\end{lstlisting}

When compiling the code, the developer's PC will contact the server at the given URL over HTTPS (verifying the server certificate via TLS) and download the software bundle. On the other end, when a library developer wants to update their library, they do so by interacting with the hosting server via HTTPS and whatever authentication the server has set up---credentials, maybe two-factor authentication, etc.

This has some good features: the server name is explicit so there is no ambiguity about packages and the decentralized nature of specifying individual URLs avoids the necessity for a central server that attracts attacks. However, this requires trusting the server hosting the library to secure the update process and distribute software honestly.

\iffalse
\subsection{Example: Rust Library Imports}
In Rust, library imports use a package manager called \ttt{cargo}. To add a library, a developer issues a command like the following:

\begin{lstlisting}[language=sh]
	cargo add rand
\end{lstlisting}

This adds another hop to the Go model: when compiling code, a developer PC will contact the central server \ttt{crates.io} to download the library. To update software, a developer will first update their code on some server such as \ttt{github.com} and then \ttt{crates.io} will contact the source server via HTTPS to update its own version. In some ways, this seems a bit more risky than the Go plan. There is now a central server managing all packages for the language, making an attractive target for attacks. The library developer is also now further away from the code that will use the library, adding an additional trusted party in between. However, this addition of a central repository has benefits as well: it provides a single place where all packages can be seen, and is a central server that makes auditing much simpler.
\fi

\subsection{More Explicit Trust: Code Signing}
In each of these approaches, if an attacker can cause the user to download
a bad package \emph{without} compromising the package developer.
In particular, if the attacker can compromise Github, it can 
cause Github to distribute malware to end users.

To prevent this attack, a library developer could
sign their software using their private key and
include the signature with their 
software package. To verify that a package is authentic,
the application developer's PC can check that the signature is
valid. 

Of course, with any signature-based plan the mechanism for public key distribution is crucially important. In the software distribution case, the only reasonable plan is likely a Trust-on-First-Use based one which accepts the first public key it sees but verifies that future software updates use that same key. This protects against an adversary taking control of, for example, the application's Github repository after the end user installs the softare once. However, key management is hard, so this is not widely used in practice.

\section{Building Binaries}
In order to run software on our computer, it is
necessary to convert the source code (which is, at
least in principle, manually auditable) into
a binary that is much more difficult to audit.
Since compiling software is computation-heavy,
most application developers typically 
compile their software and distribute the binary
to their users. \marginnote{The XCodeGhost attack
is an example of how an attacker can insert a backdoor 
in a build system and exploit it to distribute malware.}
If an attacker compromises the build server
(or is able to backdoor the compiler),
then the attacker can cause users to execute
bad code, even if the attacker does not compromise
the application developer itself.

\paragraph{Reproducible Builds.}
One promising approach to the problem of
ensuring that a binary is the faithful compilation
of a piece of software is called ``reproducible builds.''

If a build process is reproducible, the function that
turns a set of source-code files into a binary is
a deterministic function: if two different people compile
the same set of source-code files, they will get exactly
the same binary---the two will be bit-for-bit identical.
This allows anyone to \emph{audit} a build: to check that
a build server did its job correctly.
In addition, having multiple independent parties build
the same piece of software (and sign the result)
can give an end user some assurance that the build
server behaved correctly.

Implementing reproducible builds is not trivial.
Traditional compilers introduced many sources of 
non-determinism---not necessarily for any particular reason,
just for convenience.
Creating reproducible builds requires eliminating 
all of these sources of non-determinism, even across
multiple versions of the compiler.

As of today, the Go programming language now supports reproducible builds.

\subsection{Juggling multiple versions of a library}
Once a binary exists, the next step of the process
is to distribute that software to user devices.
Typically, there are many different versions of
a piece of software around.
When a user wants to install a piece of software,
they typically need to specify which version of
the software they want.

For example, in Python a user can specify 
a version of a package that they would like to 
install when they run \texttt{pip install}.
If an attacker compromises the PyPi server,
it can serve up any code it wants to a user
asking for a particular version of a library.

In contrast, in the Go programming language, 
when a users imports a package, the \texttt{go get}
software will store a hash of the downloaded code
in a file called \texttt{go.sum}.\marginnote{This
is an example of ``trust on first use'' in the context
of code installation.}
If an attacker later compromises the server 
serving the package (e.g., Github), the \texttt{go get}
command will refuse to install the package unless its
code matches the stored hash value.

\section{Installing \& Updating Software}

Once a software developer finishes writing 
an application, it builds and distributes it.
When a user installs an application---e.g., by
downloading it from a website or fetching it 
using a package manager---how does the user
know that it got the authentic version of the software?
As usual in systems design, there are many possible strategies.

\paragraph{Application Developer Signs Package (Android Apps).}
One possible option is to have application
developers sign the software that they produce.
When application developers distribute their software, 
they attach a their signature to it.
This way, it does not matter how
a user obtains the software---a user can
download an application bundle from any server and
know that it came from the developer who owns the
corresponding secret key.

When a user first installs a piece of software they
need to somehow obtain the software developer's public key.
Public-key distribution, as always, is messy: trust on 
first use is a common strategy.

Once the user has the software developer's public key,
the user can easily verify that future updates to the
software came from the same developer. (To do this,
the user can just check signatures on the updates
using the app-developer's public key.)

An important caveat is that signatures do not guarantee
freshness: once signed, a package is always valid.


\paragraph{Repository Signs Packages.}
For systems with a central repository, another plan is for the repository to sign packages. This again allows the distribution itself to be untrusted---a repository could use a CDN to distribute packages without trusting that CDN to preserve the package contents.

Many Linux package managers, such as \ttt{apt}, \ttt{pacman}, and \ttt{rpm}, use a plan like this.

\paragraph{Third-Party Validator Signs Packages.}
Yet another option that does not require a single central repository is to have a trusted validator sign packages. This involves sending the source code and package to a third party, who will then perform some inspection of the package and, if it deems a package to be worthy, provide some signature over that package that verifies that the validator thinks the package is trustworthy.

This plan is used widely in practice. On Android, there is no requirement to install apps from the Google Play Store, but Google provides a service that inspects packages and attaches these signatures if the package passes. Similarly, Windows uses a validation plan for its device drivers. 

\paragraph{Binary Transparency.}
One different plan to help involves an audit log that keeps track of all published binaries.

This helps prevent in particular targeted attacks---for example, if some adversary has a specific target in mind and compromised the distribution of the Linux kernel, they would likely be immediately noticed if they introduced a backdoor into Linux for the whole world. However, if they were able to introduce a backdoor and distribute that backdoored version only to their target, the adversary would be much more likely to evade detection. If clients check their received binary against the publicly available one before installing it, this personalized attacks can be avoided---if the attacker wants to change the binary for someone, they will need to change it for everyone.

\section{Booting the System}
In order to actually run an application, we rely on large amounts of software running on our computer, from the applications themselves to the operating system that supports them. If the operating system itself is compromised, for example, the modified OS could undermine all of the defenses we just discussed. One plan to help take care of this involves a hardcoded root of trust baked into the CPU hardware itself and a hardcoded boot ROM that cannot be updated (hence ROM, for Read Only Memory). Booting then involves several steps:
\begin{enumerate}
	\item On boot, the CPU will begin running the hardcoded Boot ROM code, which knows a hardcoded $\vk_\text{ROM}$.
	\item the Boot ROM will load the code for another layer called the \emph{boot loader}. The boot ROM will then verify that the boot loader code was signed by $\vk_\text{ROM}$. If the signature is valid, the boot ROM code will jump to begin executing the boot loader.
	\item The boot loader will load the code for the operating system and verify that it was signed by $\vk_\text{bootloader}$. If the signature checks out, the bootloader will redirect control to the operating system.
\end{enumerate}

This way, the system can verify that only boot loaders that the hardware manufacturer trusts are executed, and those boot loaders can verify that only trusted operating systems are executed.

This plan is used in many systems, from iPhone and Android to chromebooks and UEFI secure boot on PCs.
