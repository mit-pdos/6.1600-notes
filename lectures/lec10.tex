So far, we have talked about encryption systems that require
the sender and recipient to \emph{share a secret key}.
In this chapter, we discuss how the sender and recipient
can agree on a shared secret even if they only ever communicate
over an open (insecure) network.

A key-agreement scheme over keyspace $\calK$ is defined by efficient functions $(\Gen, \Derive)$:
\begin{itemize}[noitemsep]
  \item $\Gen() \to (\sk, \pk)$. The $\Gen$ algorithm 
    generates a secret key and a public key for one party.
    \marginnote{Formally, $\Gen$ also takes as input the security
    parameter.}
  \item $\Derive(\sk_A, \pk_B) \to k$. The $\Derive$ algorithm
    takes as input one party's secret key $\sk_A$ and the
    other party's public key $\pk_B$ and outputs a shared
    key $k \in \calK$.
\end{itemize}

Given this syntax, let us see how two parties can use a key-agreement 
scheme $(\Gen, \Derive)$ to agree on a shared secret:
\begin{enumerate}[noitemsep]
  \item Alice runs $(\sk_A, \pk_A) \gets \Gen()$ and sends $\pk_A$ to Bob.
  \item Bob runs $(\sk_B, \pk_B) \gets \Gen()$ and sends $\pk_B$ to Alice.
	\item Alice and Bob both then compute a key $k \in \calK$ as:
		\begin{itemize}
      \item Alice computes $k \gets \Derive(\sk_A, \pk_B)$.
      \item Bob computes $k \gets \Derive(\sk_B, \pk_A)$.
		\end{itemize}
\end{enumerate}


\paragraph{Correctness.} 
Correctness for a key-agreement scheme just says that the two parties
should always agree on the same shared secret:

\begin{definition}[Key Agreement Correctness]
We say that a key-agreement scheme is \emph{correct} if
for all 
$(\sk_A, \pk_A) \gets \Gen()$ and
$(\sk_B, \pk_B) \gets \Gen()$, it holds that: 
\[ \Derive(\sk_A, \pk_B) = \Derive(\sk_B, \pk_A). \]
\end{definition}

\paragraph{Security.}
The standard notion of security for a key-agreement scheme
only considers \emph{passive attacks}:
we consider adversaries that can view the
network traffic but cannot modify it.
In practice, we can combine key-agreement
schemes with authentication schemes (e.g., digital signatures)
to prevent active attacks by in-network adversaries.

Our security definition for key agreement says that even
if an adversary sees both parties' public keys, it should
not be able to distinguish the shared secret from random.
That is, the following probability distributions
$D_\text{real}$ and $D_\text{random}$ should be 
computationally indistinguishable:\marginnote{When we
say that two distributions are \emph{computationally indistinguishable},
we mean that if we give the adversary a sample from one or the other,
it can guess which sample it got with probability at most $1/2 + \text{``negligible''}$.}
\begin{align*}
  D_\text{real} &\deq \left\{ 
(\pk_A, \pk_B, k) \colon  \begin{aligned}
  (\sk_A, \pk_A) &\getsr \Gen()\\
  (\sk_B, \pk_B) &\getsr \Gen()\\
  k &\getsr \Derive(\sk_A, \pk_B)
\end{aligned}
\right\} \\
  D_\text{random} &\deq \left\{ (\pk_A, \pk_B, k) \colon  \begin{aligned}
  (\_, \pk_A) &\getsr \Gen()\\
  (\_, \pk_B) &\getsr \Gen()\\
  k &\getsr \calK 
\end{aligned}
  \right\}
\end{align*}


\section{Diffie-Hellman key exchange}
We now give a simple and beautiful key-exchange protocol, 
due to Diffie and Hellman.
\marginnote{So far, we have been able to construct our cryptographic entities from constructs like a PRF, which meant that we could use \textquote{unstructured} algorithms like AES to compute them. We so far only know how to construct key-exchange schemes from 
more structured problems (e.g., based on number theory).}

The version we will see uses large primes $p$ 
and $q$ a public parameters, along with a number $g \in \Z^*_p$,
called the ``generator.''
(\textbf{Warning:} The parameters $p$, $q$, and $g$ must have
some particular relation. So do not attempt to pick these parameters
yourself.)
Typically, we will have $p \approx q \approx 2^{2048}$.
\marginnote{For a prime $p$, the 
notation $\Z^*_p$ just denotes 
the non-zero integers modulo $p$.
So when we write $ab \in \Z^*_p$,
we mean $a \cdot b \bmod p$.}

The keyspace of the Diffie-Hellman scheme is $\calK = \Z^*_p$
and the algorithms are as follows:
\begin{itemize}[noitemsep]
  \item $\Gen() \to (\sk, \pk)$.
    \begin{itemize}[noitemsep]
      \item Sample $\sk \getsr \{1, \dots, q\}$.
      \item Set $\pk \gets g^{\sk} \in \Z^*_p$.
      \item Output $(\sk, \pk)$.
    \end{itemize}

  \item $\Derive(\sk_A, \pk_B) \to k$.
    \begin{itemize}[noitemsep]
      \item We have $\sk_A \in \{1, \dots, q\}$ and $\pk_B \in \Z^*_p$.
      \item Output $k \gets (\pk_B)^{\sk_A} \in \Z^*_p$.
    \end{itemize}
\end{itemize}

Before we argue correctness and security, let us consider
the computational efficiency of the scheme:

\paragraph{Efficiency.} In order for the algorithm to be useful,
Alice and Bob must be able to compute $g^x \in \Z^*_p$ efficiently,
for $x \in \{1, \dots, q \approx 2^{2048}\}$.
However, trying to compute $g^x$ where $x$ is 2048
bits long naively would certainly not be
efficient: it would require $x \approx 2^{2048}$ multiplications!
However, we can compute this exponentiation much more efficiently using
the following strategy:
\begin{itemize}[noitemsep]
  \item \textbf{Compute powers of $g$.}
        Write $\ell \gets \lceil \log_2 p \rceil$.
        Then compute $(g, g^2, g^4, g^{8}, g^{16}, \dots, g^{2^\ell})$,
        where all of these are in $\Z^*_p$.
        It is possible to compute $g^{2^{i}}$ with a single multiplication
        modulo $p$ as $g^{2^i} = (g^{2^{(i-1)}})^2 \in \Z^*_p$.
        So this step takes only $\ell = 2048$ multiplications.
        \marginnote{In many applications, the generator $g$ is fixed in advance.
        In this case, the implementation can precompute and store these powers of $g$.}
  \item \textbf{Compute exponentiation.}
        Write the bits of the exponent as $x = x_0 \cdots x_{\ell-1}$.
        Then compute:
        \[ g^x = g^{\sum_{i = 0}^{\ell-1} x_i 2^i} = \prod_{i=0}^{\ell-1} x_i(g^{2^i}) \in \Z^*_p\] 
        This step again takes only $\ell$ multiplications.
\end{itemize}


\paragraph{Correctness.}
Correctness holds since $g^{xy}=g^{yx} \in \Z^*_p$ for all $x,y \in \Z$:
\begin{align*}
  \Derive(\sk_A, \pk_B) = (\pk_B)^{\sk_A} = (\pk_A)^{\sk_B} = \Derive(\sk_B, \pk_A).
\end{align*}


\paragraph{Security.}
To argue security, we must rely on a new computational assumption:
essentially we just assume that the key-agreement scheme is secure.

\begin{definition}[Decision Diffie-Hellman (DDH) assumption]\label{defn:ddh}
The \emph{decision Diffie-Hellman assumption} states that,
for $p$, $q$, and $g$, the following distributions are 
computationally indistinguishable:\marginnote{To make the
statement fully formal, we need to let $p$, $q$, and $g$ grow
with the security parameter.}
  \begin{align*}
    D_\text{real} &\deq \{ (g, g^x, g^y, g^{xy}) \in (\Z^*_p)^4\colon x, y \,\,\,\,\,\getsr \{1, \dots, q\}\}\\
    D_\text{ideal} &\deq \{ (g, g^x, g^y, g^z) \,\,\,\in (\Z^*_p)^4 \colon x, y, z \getsr \{1, \dots, q\}\}
  \end{align*}
\end{definition}

In practice, we typically first run Diffie-Hellman key
agreement, have the two parties run the
shared secret that they get through a cryptographic hash function, 
and then use the hashed value as an encryption key.
If we model the hash function as a random oracle, security can rely
on a slightly weaker assumption---the ``computational'' Diffie-Hellman
(CDH) assumption.
The CDH assumption asserts, informally, that given $(g, g^x, g^y)$, it is
infeasible to compute $g^{xy}$.


\section{The discrete-log problem}
The DDH assumption (\cref{defn:ddh}) is no harder than the following
problem, which asserts that computing $x$ given $(g, g^x) \in (\Z_p^*)^2$
is computationally infeasible:
\begin{definition}[Discrete-log assumption]
The \emph{discrete-log assumption} states that,
for $p$, $q$, and $g$, and for all efficient 
adversaries $\calA$:
  \[ \Pr[\calA(g, g^x) = x \colon x \getsr \{1, \dots, q\} ] \leq \text{``negligible''}.\]
\end{definition}
Given an algorithm for the discrete-log problem, we can use it to solve the DDH problem.
Given a DDH challenge $(g, g^x, g^y, g^{z})$, compute the discrete log of $g^x$ and
test whether $g^z = {(g^y)}^x$.
For certain choices of $p$, $q$, and $g$, the best known algorithm for the DDH problem
is to first solve the discrete-log problem in $\Z^*_p$.

How hard it to solve the discrete-log problem then?
\begin{enumerate}
	\item The most basic attack is to enumerate all $p$ possible values of $x$ and check whether the corresponding $g^x$ matches. This will take time $p \approx 2^{2048}$.
	\item There is a slightly more clever algorithm, called ``Baby Step Giant Step,'' that is able to compute $x$ in time $\sqrt{p}$.
  \item In $\Z^*_p$,
    a much better attack is the Number Field Sieve.
    This algorithm is able to compute $x$ in (roughly) time $\exp((\log p)^{1/3} (\log \log p)^{2/3})$---sub-exponential time!
    \marginnote{An attack that runs in time $\sqrt p$ runs in time $2^{\frac 12 \log p}$. 
    In contrast, the Number Field Sieve runs in time roughly $2^{\sqrt[3]{\log p}}$. This is much much much faster than the $\sqrt p$-time algorithms,
    since the exponent grows much more slowly.}
    The existence of this attack is the reason we require $p$ to be $2048$ bits long to 
    get $128$-bit security.
\end{enumerate}






\section{Generalizations of Diffie-Hellman}

We have described the Diffie-Hellman protocol in terms
of $\Z^*_p$---multiplication of integers modulo $p$.
In particular, the protocol uses a \emph{set} of elements (here, $\Z^*_p$)
and a \emph{binary operation on elements} (here, multiplication modulo $p$).
There is a natural generalization of the Diffie-Hellman
protocol that works with other sets of elements and
binary operations that operate on them.
\marginnote{More precisely, we can define Diffie-Hellman
key exchange with respect to any \emph{finite cyclic group}---a
concept from abstract algebra.
A cyclic group just consists of a \emph{set} of elements
and a \emph{binary operation on elements}.
The set and operation need to satisfy certain mathematical
properties---associativity, etc.

Given a group $\G$, we can then 
define a discrete-log assumption on $\G$ and whenever
discrete-log is hard on $\G$, we can use $\G$ to construct
cryptosystems.
}

The most widely used version of the Diffie-Hellman protocol today
uses \emph{elliptic-curve groups}.
The set of elements in an elliptic-curve group is a set of points $(x,y) \in \Z^2$
in two-dimensional space, 
where $0 \leq x, y \leq p$ for some prime $p \approx 2^{256}$.
The binary operation on elements is some geometric operation
on two points that yields a third point.
Even though the underlying objects are now points instead of integers
modulo $p$, the Diffie-Hellman protocol looks exactly the same in this setting.

The appeal of elliptic-curve cryptosystems is that the best known discrete-log
algorithm on certain elliptic-curve groups is Baby Step Giant Step, which runs
in $\sqrt p$ time.
So, we can use elliptic-curve groups of size $2^{256}$ and the Diffie-Hellman
public keys take only $\approx 256$ bits to represent.
In contrast, when working in $\Z^*_p$, we need to work modulo a prime $p \approx 2^{2048}$
to defeat the Number Field Sieve attack, so Diffie-Hellman public keys in
this setting take $\approx 2048$ bits to represent.



\iffalse
\section{Background: Group theory}
The simplest and most commonly used key-exchange protocol uses
a mathematical concept called a \emph{group}.
A group $\G$ is a set of elements along with an operator
``$\cdot$'' that takes two group elements $g, h \in \G$
and returns a third group element $g \cdot h \in \G$.
A group 


\section{Finite Groups}
In the last section, we defined Diffie-Helman in terms of the \emph{finite group} $\Z_p^*$. However, this scheme can work more generally over any finite group as follows:

\begin{compactenum}
	\item Choose some $g \in G$ that is a \emph{generator} of $G$.
	\item Alice samples $a \gets \abs{G}$ and sends $g^a$ to Bob.
	\item Bob samples $b \gets \abs{G}$ and sends $g^b$ to Alice.
	\item The key is then $g^{ab}$.
\end{compactenum}

The CDH assumption is that it is hard to compute $g^{ab}$ given $g^a$ and $g^b$.

\begin{definition}[Finite Group - Generator]
	A \emph{generator} of a finite group $G$ is an element $g$ such that $G = \set{g, g^2, \ldots, g^q}$ for $q = \abs{G}$. Note that generators are not necessarily easy to find. However, for groups of prime order, every element except 1 is guaranteed to be a generator of the group.
\end{definition}

\begin{definition}[Safe Prime]
	A prime $p$ is a \emph{safe prime} iff $\tfrac{p-1}{2}$ is prime.
\end{definition}

\begin{definition}[Finite Group]
	A finite group consists of a finite set $G$ and binary operation $\cdot$ with the following properties:
	\begin{compactenum}
		\item \textbf{Closure}: $\forall a, b \in G, a \cdot b \in G$
		\item \textbf{Identity}: $\exists 1 \in G$  s.t. $\forall a \in G, a\cdot 1 = 1\cdot a = a$.
		\item \textbf{Associativity}: $\forall a, b, c \in G, a\cdot (b \cdot c) = (a \cdot b) \cdot c$
		\item \textbf{Inverse}: $\forall a \in G, \exists a^{-1} \in G$ s.t. $a\cdot a^{-1} = a^{-1}\cdot a = 1$.
	\end{compactenum}
\end{definition}

The group that we discussed last section, $\Z_p^*$ uses the set of elements $\set{1, \ldots, p-1}$ and defines $\cdot$ as multiplication modulo $p$. The size of $\Z_p^*$ is $\abs{\Z_p^*} = p-1$---even since if $p$ is prime then it is odd. We also discussed the quadratic residual, which is defined with $\set{a^2: a \in \Z_p^*}$ and again uses multiplication modulo $p$. This groups is of size $\abs{QR} = \tfrac{p-1}{2}$. Importantly, $\tfrac{p-1}{2}$ may be prime if $p$ is a \emph{safe prime}.

There are many other groups besides these!
\begin{compactitem}
\item $\Z_p$: $\set{0, 1, \ldots, p-1}$, with the operator as addition modulo $p$.
\item $n \times n$ full rank matrix defined modulo $p$.
\item Elliptic Curve groups are some of the most useful groups in modern cryptography.
\end{compactitem}
\fi
