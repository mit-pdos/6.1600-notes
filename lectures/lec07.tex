\chapter{Public-key Infrastructure}

In the last chapter, we discussed digital
signatures, which allow us to authenticate messages
without a shared secret.
For example, if I have the public signature-verification key 
of the university dean, I can verify that signed emails from 
the dean really came from her and not from someone pretending to be her.
But to verify the signature on the dean's message, I need to know her
signature-verification key $\vk$.
How can I (the recipient) obtain this verification key without
a secure channel to the dean (the sender)?

Unfortunately, there are no perfect solutions to this problem.
In this section, we will discuss some of the approaches that
we use in practice.

\section{Public-key infrastructure}

The goal of a public-key infrastructure is to facilitate
the mapping of \textbf{human-intelligible names} to
\textbf{signature-verification keys}.
Examples of human-intelligible names that we map to keys
are: email addresses, domain names, legal entities, phone numbers, and
usernames (e.g., within a company).

We can think of the public-key infrastructure as implementing
the following (grossly simplified) API:
\[ \mathsf{IsKeyFor}(\vk, \texttt{name}) \to \zo.\]
That is, given a verification key $\vk$ and a name $\texttt{name}$, 
the public-key infrastructure gives a way to check whether this mapping
is valid.

% TODO: exmaples: chisel, meet in person, trusted source, blockchain

We now discuss some ways to implement this API.

\section{Option 1: Use verification keys as names}
One option is to just refer to everyone by the bytes of
their signature-verification key.
This way, there is no need to do a messy name-to-verification-key 
translation at all.

This is not practical for humans generally: it would be difficult
to remember your friends' names if you had to call them by 
random 32-byte strings!
However, some digital services such as Bitcoin indeed use verification
keys as identities: when you want to transfer Bitcoin to someone,
you send the coins to an account identified by their public key.
The public key \emph{is} name of that account.

Using keys as names has a two major problems:
\begin{itemize}
	\item Verification keys are hard to remember. Things like email addresses, domain names, kerberos usernames, phone numbers, and so on, are much easier for humans to remember.
	\item There is no way to update the name-to-key mapping. If you lose the secret key associated with your name/account, there is no way to ``update'' the key to a new value. In practice, people lose their secret keys all the time, so supporting key updates is critical in most systems.
\end{itemize}

% TODO: this introduction does not flow very well
% it might be good to have a section elaborating on the need to change keys a bit more.

\section{Trust on first use (TOFU)}
Another strategy is to avoid having any global mapping from names to verification keys.
Instead, a client can just accept the first verification key 
that it sees associated with a given name.
The secure shell system (\ttt{SSH}) uses TOFU for key management by default.

In particular, the key-validation logic looks like this:

\begin{framed}
\noindent
$\mathsf{keymap} \gets \{ \}$.

\medskip
\noindent
$\mathsf{IsKeyFor}(\vk, \texttt{name}):$ 
\begin{compactitem}
\item If $\mathsf{keymap}[\texttt{name}]$ is undefined:
      \begin{compactitem}
      \item Set $\mathsf{keymap}[\texttt{name}] \gets \vk$.
      \item Return true..
      \end{compactitem}

\item Else: Return $\mathsf{keymap}[\texttt{name}] == \vk$.
\end{compactitem}
\end{framed}

That is, the client will accept the \emph{first} verification key 
it sees associated with a particular name.
Later on, the client will only accept the same verification key
for that name.

TOFU is very simple to implement and provides a meaningful security
guarantees. 
There are two drawbacks:
\begin{itemize}
  \item If the first key that client receives for a particular name 
        is incorrect/attacker-generated, the attacker can forge signatures
        under that name.
  \item It is not clear with TOFU how to handle key updates. In most systems that use TOFU, whenever the sender's key changes, the system notifies the user and allows them to accept or reject the new key. The burden is then on the user to figure out whether the sender really did change their signing keypair, or whether there is an attack in progress.
\end{itemize}

\section{Certificates}
Another option is to rely on a few parties to
manage the mapping of names to public keys.
These entities are called \emph{Certificate Authorities} (CAs).
Your operating system and web browser typically come bundled with a
set of roughly 100 public signature-verification keys, owned by 
each of 100 CAs.

Whenever the owner of website \ttt{example.com}, for example, generates 
a new public key $\vk_\ttt{example.com}$, the website owner can ask a 
certificate authority to certify that $\vk_\ttt{example.com}$ really belongs to \ttt{example.com}.
The certificate authority does this by signing the
pair $(\ttt{example.com}, \vk_\ttt{example.com})$
using its own signing keypair $\vk_\text{CA}$
to generate a signature $\sigma_\text{CA}$.
This signed attestation $(\ttt{example.com},\allowbreak \vk_\ttt{example.com},\allowbreak \sigma_\text{CA})$
is called a \textit{certificate}.
\marginnote{In practice the structure of certificates is much more complicated
than we are showing here, and include all sorts of additional metadata. But the 
basic idea is the same as we describe here.}

When a client connects to \ttt{example.com}, the server at \ttt{example.com}
will supply the client with the certificate $(\ttt{example.com},\allowbreak \vk_\ttt{example.com},\allowbreak\sigma_\text{CA})$.
As long as this certificate was signed by a CA that the client trusts (i.e., 
a CA for which the client has a verification key),
the client can validate the certificate and conclude that 
the verification key $\vk_\ttt{example.com}$ really belongs to \ttt{example.com}.

In pseudocode the logic for verifying certificates looks like this:
\begin{framed}
\noindent
$\mathsf{caKeys} \gets \{\vk_\text{Verisign}, \vk_\text{Let's Encrypt}, \dots \}$.

\medskip
\noindent
$\mathsf{IsKeyFor}((\vk, \sigma), \texttt{name}):$ 
\begin{compactitem}
\item For each $\vk_\text{CA}$ in $\mathsf{caKeys}$:
      \begin{compactitem}
      \item If $\Ver(\vk_\text{CA}, (\texttt{name}, \vk), \sigma) = 1$: Return true.
      \end{compactitem}
\item Return false.
\end{compactitem}
\end{framed}

A very nice feature of certificate-based public-key infrastructure is that
the client does not need to communicate with the CA to validate a 
name-to-key mapping. The client only needs to perform one signature-verification
check.
\marginnote{In order to use TLS on a website you own, you need to convince one of the certificate authorities to give you a certificate---i.e., to sign your $(\ttt{name}, \vk)$ pair. 
To do so, the CA will have some protocol to follow---typically, you will send your $(\ttt{name}, \vk)$ pair to the CA, who will then ask you to verify that you own the name somehow. This is called \textit{domain validation}.
In the case of web certificates, the CA may verify ownership by requiring you to upload a file file to your server, to add a new DNS record with a random value, or something similar.
Once the CA is convinced that you own the domain, the CA will reply with a certificate: a signature over the tuple $(\ttt{name}, \vk)$. This $(\ttt{name}, \vk, \sig_{\text{CA}})$.
}

Certificates in practice works quite well:
\begin{itemize}[noitemsep]
	\item The client only needs to store $\approx 100$ CA verification keys,
        and yet the client can validate the name-to-key mappings for millions of websites.
	\item A client can choose which CAs to trust 
        (though in practice, clients typically delegate this decision to software vendors).
	\item The client never needs to interact with the CA.
\end{itemize}

However, certificates still have some drawbacks:
\begin{itemize}[noitemsep]
  \item If an attacker compromises \emph{any} CA, they can generate certificates for any domain.
	\item Certificate authorities often perform quite minimal validation of domain ownership.
	\item If a server's private key gets stolen, 
        there is no great plan for \emph{revoking} or updating a name-to-key mapping.
\end{itemize}

\subsection{Domain Validation}

Domain Validated (DV) certificates are quite common. CA validation uses a simple technical check for domain ownership. For example, the CA asks the requester amazon.com to put a nonce in a file on an amazon.com server. Then, the CA retrieves that file, and checks the nonce. The (reasonable) assumption is that only the server owner could have created such a file. Alternately, the CA could ask the requester to create certain Domain Name Services (DNS) records under amazon.com, e.g., acme-challenge.amazon.com.

This check is often automated, and is fast, convenient, cheap. An example DV CA is {\bf Let's Encrypt}. It is easy to use and free; the requester doesn't have to be a real company.

DV certificates only have low-level DNS ownership guarantees: "CA validated that owner of certificate owns amazon.com". However, the guarantee closely matches what CAs can be expected to validate in real life. 

A standard protocol for validating domain ownership is ACME\footnote{https://tools.ietf.org/html/rfc8555\#section-8.3}.

\subsection{Revocation}
In many cases, a CA will want to delete or change a name-to-key mapping.
This process is called \emph{certificate revocation}.
There are several possible reasons for this:
\begin{itemize}[noitemsep]
	\item The owner of a verification key may have their corresponding secret key
        be lost or stolen.
	\item A company may want to rotate keys, 
        for example to update to a new cryptographic algorithm.
	\item A website may go out of business and another entity buys their domain name.
	\item Software bugs may lead a user to generate an
        insecure keypair that they later want to revoke.\autocite{yilek2009private,nemec2017return}
\end{itemize}

In a scheme that uses certificates, this seems like a hard problem: since there is no interaction with the CA to verify a certificate, there is no way for a CA to \textquote{take back} a certificate. There are again no excellent solutions to this, but there are a few strategies used in practice.

\paragraph{Expiration Dates}
One pragmatic way to handle revocation is to add
an expiration date to each generated
certificate---if this expiration date has passed,
the client will reject the certificate. This way,
a server will need to re-authenticate to the CA
that they own the name that they claim to own
periodically. So, for example, an attacker that steals
a website's secret key will only be able to use it 
until the certificate expires.
In practice, certificates used on the Internet
typically expiration dates between 90 days 
and 1-2 years.

\paragraph{Software Updates}
Another solution is for the browser (or client, more generally)
to maintain a list of revoked certificates.
On every connection, the browser will check whether the provided certificate is in this local revocation list and refuse it if so.
Since browsers today check for updates very frequently, this strategy can respond to a stolen secret key quickly. However, there is a large storage cost since now every browser needs to store this (potentially large) list of revoked certificates.

\paragraph{CA Revocation List}
To avoid depending on browser manufacturers to update this revocation list, another strategy is to ask the CA for it directly. One way to do this is similar to the above: periodically query the CA to download its updated revocation list and check that each certificate is not in this list. This method has fallen out of favor, in part because clients (e.g., behind corporate firewalls) cannot connect directly to the CAs to download these revocation lists.

