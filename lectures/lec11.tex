\chapter{Public-Key Encryption}

So far, all of the encryption schemes we have seen
require the sender and recipient to 
\emph{share a secret encryption key}.
In this section, we show that it is possible for
a sender to encrypt a message to a recipient even
if the two parties do not share a secret key.

In particular, in ``symmetric-key'' or ``secret-key'' encryption
schemes we have seen so far, the sender and recipient use the \emph{same key}
to encrypt and decrypt messages.
In the ``asymmetric-key'' or ``public-key'' encryption schemes
we consider now, the sender and recipient use \emph{different keys}
to encrypt and decrypt messages.

\section{Defining Public-Key Encryption}
The definition for a public-key encryption scheme will be similar to the 
definitions we saw for symmetric-key encryption:

\begin{definition}[Public-Key Encryption Scheme]
	A public-key encryption scheme over message space $\calM$ 
  consists of three efficient algorithms $(\Gen, \Enc, \Dec)$:
	\begin{compactitem}
    \item $\Gen() \to (\sk, \pk)$: Generates a keypair with secret key $\sk$ and public key $\pk$.
    \item $\Enc(\pk, m) \to c$: Uses public key $\pk$ to encrypts a message $m \in \calM$ 
            to a ciphertext $c$.
    \item $\Dec(\sk, c) \to m$: Uses secret key $\sk$ to decrypt ciphertext $c$ into message $m \in \calM$.
	\end{compactitem}
\end{definition}

\begin{definition}[Public-Key Encryption - Correctness]
For all keypairs $(\sk, \pk) \gets \Gen()$ and for all messages $m \in \calM$,
it holds that $$\Dec(\sk, \Enc(\pk, m)) = m.$$
\end{definition}

\begin{definition}[Public-Key Encryption - Security against chosen-plaintext attacks (Weak)]
  \marginnote{The literature sometimes calls security against chosen-plaintext attacks (``CPA security'')
  \emph{semantic security}.}
  A public-key encryption scheme $(\Gen, \Enc, \Dec)$ is \emph{secure against chosen-plaintext attacks}
  if all efficient adversaries $\calA$ 
  win the following game with probability $\leq \tfrac{1}{2} + \negl$:

	\begin{compactitem}
  \item The challenger generates $(\sk, \pk) \gets \Gen()$ and $b \getsr \bin$ and sends 
        the public key $\pk$ to $\A$.
  \item The adversary $\A$ sends $m_0, m_1 \in \calM$ to the challenger,
        where $\abs{m_0} = \abs{m_1}$.
  \item The challenger responds with $\Enc(\pk, m_b)$		
  \item The adversary outputs $b'$ and wins if $b' = b$.
	\end{compactitem}
\end{definition}
Note that since this is a public-key encryption
scheme, the adversary can use the public key to
generate encryptions of any message of their choice. 
As in the secret-key setting, here it is crucial that
the encryption algorithm be randomized---otherwise
two encryptions of the same message are identical
and the attacker can break CPA security.

For symmetric-key encryption, we also defined
a stronger notion of security that we called
security against chosen-ciphertext attacks (CCA security).
We can extend this definition of CCA security
to the public-key setting.

This security definition asserts that the attacker should not be
able to distinguish the encryption $c^*$ of two messages of its choice,
even if the attacker can obtain decryptions of any ciphertext except
$c^*$.
This is a very strong notion of security (since the security definition
gives the attacker a lot of power) and it is the notion of security
that we typically demand in practice.

\begin{definition}[Public-Key Encryption - Security against chosen-ciphertext attacks (Strong)]
  A public-key encryption scheme $(\Gen, \Enc, \Dec)$ is \emph{secure against chosen-ciphertext attacks}
  if all efficient adversaries $\calA$ 
  win the following game with probability $\leq \tfrac{1}{2} + \negl$:
\begin{compactitem}
\item The challenger generates $(\sk, \pk) \gets \Gen()$ and $b \getsr \bin$ and sends 
  the public key $\pk$ to the adversary $\A$.
\item The adversary may make polynomially many \emph{decryption queries}:
    \begin{compactitem}
    \item The adversary $\A$ sends ciphertext $c$ to the challenger.
    \item The challenger responds with $m \gets \Dec(\sk, c)$		
    \end{compactitem}
\item At some point, the adversary sends a pair of messages $m_0, m_1 \in \calM$
  to the challenger, where $\abs{m_0}=\abs{m_1}$.
\item The challenger returns $c^* \gets \Enc(\pk, m_b)$.
\item The adversary can continue to make decryption queries, provided that
      it never asks the challenger to decrypt the ciphertext $c^*$.
\item The adversary outputs $b'$ and wins if $b' = b$.
\end{compactitem}
\end{definition}

\section{ElGamal Encryption Scheme}
In public-key encryption, we want a sender to be able to encrypt a message
to a recipient 
The goal of public-key encryption is to perform encryption without a shared secret key.

ElGamal's encryption scheme essentially uses Diffie-Hellman key exchange to allow
the sender and recipient to agree on a shared secret key $k$, and then has
the sender encrypt her message with a symmetric-key cryptosystem under key $k$.
\marginnote{ElGamal's scheme was not the first
public-key encryption scheme. The first scheme,
RSA, was much more complicated despite
Diffie-Hellman already having been published.}

\begin{definition}[Hashed ElGamal Encryption]
Let $p$, $q$, and $g$ be integers of the sort we use for Diffie-Hellman
key exchange (\cref{sec:dh}).
In particular, $p \approx q \approx 2^{2048}$
and $g \in \Z^*_p$.
  \marginnote{In practice, we typically use elliptic-curve groups to 
  instantiate ElGamal encryption, instead of $\Z^*_p$.
  But the general principle is exactly the same.}

Let $H: \Z^*_p \rightarrow \bin^*$ be a hash function (modelled as a random oracle). Let $(\Enc', \Dec')$ 
be a symmetric-key authenticated-encryption scheme. Then define:

\begin{compactitem}
\item $\Gen() \to (\sk, \pk)$:
    \begin{itemize}[noitemsep]
          \item Choose $a \gets \{1, \dots, q\}$.
          \item Compute $A \gets g^a \in \Z^*_p$.
          \item Output $(\sk, \pk) \gets (a, A)$.	
    \end{itemize}
  \item $\Enc(\pk, m) \to c$:
    \begin{itemize}[noitemsep]
      \item Choose $r \gets \{1, \dots, q\}$.
      \item Compute $R \gets g^r \in \Z^*_p$.
      \item Compute $k \gets H(\pk^r \in \Z^*_p)$.
      \item Output $c \gets (R, \Enc'(k, m)$.
    \end{itemize}

  \item $\Dec(\sk, c) \to m$:
    \begin{itemize}[noitemsep]
      \item Parse $(R, c') \gets c$.
      \item Compute $k \gets H(R^a \in \Z^*_p)$.
      \item Output $m \gets \Dec'(k, c')$.
    \end{itemize}
\end{compactitem}
\end{definition}

\subsection{Performance}
The performance of this scheme is limited by the exponentiations---the symmetric encryption scheme is quite fast (gigabytes per second), but a single exponentiation can take a millisecond on modern processors. For encryption, this scheme requires two exponentiations ($g^r$ and $\pk^r$). Decryption requires one exponentiation $(R^\sk)$.
To speed up the encryption routine, we can precompute powers of $g$: $g^2, g^4, g^8, g^{16}, \dots$, which saves a factor of two in exponentiations.
\marginnote{Hashed ElGamal encryption is one of the most common public-key encryption
schemes used today.}

\subsection{Security}
If we model the hash function $H$ as a random oracle, 
we can prove security of ElGamal encryption from 
(a) the computational Diffie-Hellman assumption (\cref{defn:cdh}) and
(b) the CCA security of the underlying authenticated-encryption
scheme $(\Enc', \Dec')$.

