\chapter{Isolation}

In many settings when building systems, it is
useful to \emph{isolate} different components of
a computer system.
For example, cloud providers such as Amazon Web Services' EC2 run multiple virtual machines,
your phone runs several apps, and your browser runs many sites together. 
If one of these virtual machines (or apps or websites) 
is malicious or buggy, these platforms would like to protect the buggy
component from interfering with the execution of other code in the system. 
We often refer to separate isolated components as running in
separate \emph{isolation domains}.

Ideally, we could have each isolation domain run on 
separate physical computer.
If this were the case, (modulo physical side-channel attacks)
one domain would clearly not be able to touch another's state.
Unfortunately, the cost of this physical ``air gap'' isolation is 
far too high for most practical applications.
\marginnote{Computer systems for high-stakes 
applications (e.g., classified government data)
do in fact use this type of ``air gap'' isolation.}



So, when creating isolation methods, we aim to make it appear as if each domain is running on a separate computer but to do so all on the same computer.

\section{Defining Isolation}
In order to define isolation, we will think about two key properties: integrity and confidentiality.

\subsection{(Weak) Integrity}
One property that we would like an isolation scheme
to provide is \emph{integrity}:
one domain cannot modifying the state of another
domain. To formalize this, let's consider two
domains running on a single host, an adversary
domain $A$ and a victim domain $V$. We would like
to guarantee that $A$ cannot change the execution
of $V$. We can define something in terms of the
\emph{state} of each domain, $S_A$ and $S_V$. For
any pair of starting states $(S_A, S_V)$, after
running $A$, we would like the new pair of states
to be $(S_A', S_V)$:
\[ (S_A, S_V) \xrightarrow{\text{run} A} (S_A', S_V) \]

\subsection{(Weak) Confidentiality}
We would also like \emph{confidentiality}: an adversarial domain should
not be able to read a data from any other domain.
We can formalize this by considering two worlds,
each with a \emph{different} victim state.
After running $A$ in each of these worlds, we would like
the resulting $S_A'$ to be identical.
That is, for all pairs of victim states $S_V^1, S_V^2$,
running $(S_A, S_V^1)$ results in the same adversarial
state as running $(S_A, S_V^2)$:
\[ (S_A, S_V^1) \xrightarrow{\text{run} A} (S_A', -) \]
\[ (S_A, S_V^2) \xrightarrow{\text{run} A} (S_A', -) \]
This definition of confidentiality is often called \textquote{non-leakage}.

\subsection{Non-interference: Strong confidentiality and integrity}
In our definitions of confidentiality and integrity so far, 
only the adversary domain runs.
In a real system, both the adversary and the victim domain will run concurrently.
Ideally, we would like to ensure that our confidentiality and integrity
properties hold under interleaved execution of the adversary and victim.

To achieve this stronger notion of security, we
can include an interleaving of $A$ and $V$ in one
world---we would like for the resulting $S_A'$ to
be identical whether or not $V$ runs.

\[ (S_A, S_V) \xrightarrow[A, V, V, A, A]{\text{run}\, A, V} (S_A', -) \]
\[ (S_A, S_V) \xrightarrow[A, A, \ldots, A]{\text{run only}\, A} (S_A', -) \]

This definition is often called \emph{non-interference}. We can similarly strengthen our definition of integrity by requiring that $S_V'$ is identical after running $V$ whether or not $A$ is run.
\[ (S_A, S_V) \xrightarrow[A, V, V, A, A]{\text{run}\, A, V} (-, S'_V) \]
\[ (S_A, S_V) \xrightarrow[A, A, \ldots, A]{\text{run only}\, V} (-, S'_V) \]

\subsection{Non-interference is difficult to achieve}

To achieve a non-interference style of isolation, 
an adversarial process $A$ must not be able to 
determine whether there is a victim $V$ process running
alongside it concurrently.
The challenge, though, is that whenever the adversary
and victim \emph{share limited resources}---such as CPU, RAM,
network bandwidth, hard disk space, etc.---it is almost
always possible for the adversary to determine whether
there is a victim process running concurrently.
\marginnote{The information leakage across isolation boundaries
as a results of resource contention is one type
of \emph{side channel} or \emph{covert channel}.
There is a vast literature on how to construct
and exploit various types of side channels that
leak information from a victim to an adversary.
}

\paragraph{Example: Memory Allocation.} A real system will have some bound on the amount of memory available to it. After this memory is used, the system will be unable to allocate any additional memory. Consider a system with 16GB of memory and a victim process that allocates memory based on the value of some secret:

\begin{lstlisting}
int secret;
malloc(secret);
\end{lstlisting}

An adversary could repeatedly try to allocate memory until the system tells them they cannot. By keeping track of the amount of memory they were able to allocate, the adversary can learn the secret: if the adversary is able to allocate 15GB, the adversary will know that the secret is $2^{27}$, as the victim must have allocated 1GB.

\paragraph{Example: Execution Time.} Consider another victim that runs some computation that takes a variable amount of time to finish depending on the value of a secret. By keeping track of how long the adversary takes to finish, the adversary can learn how much execution time the victim running on the same system takes to finish. Using this information, the adversary may be able to learn information about the secret. This type of information transfer are often called \textquote{timing channels}, and can be quite tricky to work with.

\medskip

There are effectively three ways to deal with the fact
that non-interference is generally impossible to 
achieve with shared limited resources:
\begin{itemize}
\item \textbf{Strictly partition resources to prevent contention.}
      Each isolation domain could run on a separate physical machine,
      or we can provision the resources on a machine are partitioned
      in such a way that there is never contention for resources
      between isolation domains.

\item \textbf{Prevent isolation domains from detecting resource contention.}
      Another (more practical) approach is to restrict the types
      of programs that can runs in such a way that prevents
      the programs in the system from detecting contention in 
      shared resources.
      For example, if all programs in an isolated system are deterministic
      functions with no access to the outside world---no system calls,
      no networking, etc.---then programs may not be able to \emph{detect}
      resource contention when it exists.
      In most implementations of isolation (e.g., virtual machines in
      a cloud environment), isolation domains absolutely need access to the
      outside world, so this approach is rarely useful.

\item \textbf{Give up on non-interference.}
      Most isolation mechanisms opt for this solution.
      Rather than trying to achieve strict non-interference,
      we aim for some ``good enough'' notion of isolation.
      Linux, for example, does not attempt to achieve
      strict non-interference between processes running
      on the same physical machine.

\end{itemize}

\section{Implementing Isolation}
In principle, implementing isolation in a system involves three main steps:
\begin{enumerate}
	\item identify the state for each domain,
	\item identify operations that access state, and
	\item ensure that state-modifying operations 
        can only read/write the state within
        an isolation domain.
\end{enumerate}

The challenge is performance.
An isolation mechanism will have to perform checks to
ensure that components in one isolate cannot influence
another.
The game in isolation is to provide strong isolation 
at the minimum possible cost.

We will start by considering simpler isolation mechanisms
and then look at more sophisticated ones.

\subsection{Emulation}

One simple way to implement isolation is to have
an interpreter that executes isolated programs
(e.g., x86 programs).
The interpreter inspects each opcode in the program
one at a time, and then implements the operations
on the isolated state that the opcode indicates.
While processing each opcode, the interpreter
enforces checks on the isolated program to 
ensure that it can only modify its local state.

Some JavaScript engines (``runtimes'') in web browsers use emulation
for isolation. 
The Python interpreter is another example of emulation-based isolation.
The runtime for these languages is
designed such that the code can access only memory
that belongs to the domain.

\paragraph{Benefits}
Emulation can be simple to implement and provides ``good enough''
isolation for many applications.

\paragraph{Downsides}
Emulation can be slow: to execute each logical opcode in the
emulated program, the emulator may have to run a large
number of physical instructions.
Emulation can be inflexible: 
emulated programs may not be able to take advantage
of special-purpose hardware, unless the emulator
explicitly grants access to these devices to emulated
programs.


\subsection{Time Multiplexing}

Another effective isolation strategy is
to only allow the code from one isolation
domain to run on the hardware at once.
When the isolation mechanism switches from 
one isolation domain to another, it writes
the state of the hardware (e.g., registers, RAM)
to storage, clears the state of the hardware
(e.g., zeros the contents of memory), and 
loads the next program to run into the machine.

For example, gaming consoles implement this
form of isolation: one game runs at a time
and has almost full control of the hardware
during this time.
The state of the running game cannot tamper with
the state of a non-running game.

The security of this isolation scheme requires
the isolation mechanism to protect the stored
state of each isolation domain, and to somehow
protect the code that switches between them.

\paragraph{Benefits}
Time multiplexing is relatively simple to
implement, and it can give programs in 
each isolation domain almost full control
of the hardware (e.g., graphics hardware
in game consoles)..

\paragraph{Downsides}
There can be a high cost of switching 
execution between between isolation domains.
Since the isolation mechanism clears the
state of the hardware during context switches, 
storing and restoring the state can be costly.

\subsection{Translation (Naming)}
Translation is another common isolation mechanism 
in computer systems.
In a system using translation for isolation, 
an isolated program may not access hardware resources
(such as memory or files)
directly.
Isolated programs can only access resources 
via pointers controlled
by the isolation mechanism.
By construction, each process in an isolation domain can
only name resources inside of its isolated context.

The canonical example of naming/translation for
isolation is \emph{virtual memory}.
In a system using virtual memory, the isolated
domains cannot read/write physical memory directly---they
can only read/write to virtual memory addresses..

To prevent one isolated program from accessing another's memory, 
the isolation mechanism ensures that the valid virtual addresses
in each domain point to a separate portion of physical memory.
Modern CPUs have special support for implementing virtual 
memory to make the virtual-address translation as fast as possible.

Other examples of naming/translation isolation in computer
systems are: file descriptors in Linux 
and virtual LANs in networking.

\subsection{Example: Isolation in Virtual Machines}
When we run several virtual machines on one
physical machine, we want each virtual machine to run as if
it had its own physical CPU, memory, and devices, but we
want to run all of these virtual machines on a single
physical machine.
For performance, we
would like to run instructions from the VM
directly on the host CPU---but we need to make
sure, for example, that the VM does not access
memory belonging to another VM. To achieve
isolation and good performance, systems today use
several effective techniques.

Here is how a virtual-machine monitor handles the three questions that 
an isolation mechanism must answer:

\begin{enumerate}
  \item \textbf{What is the isolation domain's state?}
        A few important pieces are:
        \begin{itemize}
          \item the contents of memory,
          \item the values in the CPU registers,
          \item the data on disk.
        \end{itemize}

\item \textbf{What operations can a program perform on isolated state?}
        Virtual-machine monitors need to handle: 
        \begin{itemize}
          \item CPU instructions that modify register states ,
          \item CPU instructions that modify memory, and
          \item operations to read and write devices.
        \end{itemize}


\item \textbf{How does the virtual-machine monitor ensure isolation?}
        \begin{itemize}
          \item To handle updates to the register state, the monitor
                uses \emph{time multiplexing}: one virtual machine
                runs on the CPU at a time and controls the
                CPU's registers.
          \item To handle accesses to memory, the monitor uses
                \emph{naming/translation} via virtual memory.
          \item To handle device operations, the monitor uses 
                \emph{emulation}.
                When a virtual machine executes instructions that
                would cause device I/O, the CPU jumps into some dedicated
                code in the monitor that emulates these device operations.
                This is slow, but since device I/O is usually costly
                anyways, the overhead is isolation is tolerable.
        \end{itemize}
  
\end{enumerate}


\subsection{Software interposition}

A final isolation technique that we will discuss
is \emph{software interposition}.
Say that an isolated program wants to run the following code:
\begin{lstlisting}
var a = b[c];
var f = ....;
f();
\end{lstlisting}

When using software isolation, a compiler can insert checks
into this isolated program to make sure that the memory access
\texttt{b[c]} does not access out-of-bounds memory:
\begin{lstlisting}
if c >= b.size: error;
load b+c -> a             (***)
\end{lstlisting}
After inserting these checks,
the isolated code can run directly on hardware,
since the checks ensure that the isolated code cannot touch 
any state outside of its isolated context.

A central challenge of using software interposition is handling
function calls.
When calling the function \texttt{f} in the code snippet above,
the hardware might execute an instruction like:
\begin{lstlisting}
jump *f
\end{lstlisting}
which would execute the code at the location stored in 
memory location \texttt{*f}.

If an isolated program can set the value in \texttt{*f}
to the location \texttt{(***)} in the snippet above,
the isolated program can skip the safety checks on the 
memory accesses.


